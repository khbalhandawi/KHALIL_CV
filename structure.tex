%%%%%%%%%%%%%%%%%%%%%%%%%%%%%%%%%%%%%%%%%
% Freeman Curriculum Vitae
% Structure Specification File
% Version 1.0 (19/3/2018)
%
% This template originates from:
% http://www.LaTeXTemplates.com
%
% Authors:
% Vel (vel@LaTeXTemplates.com)
% Alessandro Plasmati
%
% License:
% CC BY-NC-SA 3.0 (http://creativecommons.org/licenses/by-nc-sa/3.0/)
% 
%%%%%%%%%%%%%%%%%%%%%%%%%%%%%%%%%%%%%%%%%

%----------------------------------------------------------------------------------------
%	PACKAGES AND OTHER DOCUMENT CONFIGURATIONS
%----------------------------------------------------------------------------------------

\usepackage{etoolbox} % Required for conditional statements

\setlength\parindent{0pt} % Stop paragraph indentation

\usepackage{supertabular} % Required for the supertabular environment which allows tables to span multiple pages so sections with tables correctly wrap across pages

\usepackage{array} % for custom alignment if supertabular columns

%----------------------------------------------------------------------------------------
%	DOCUMENT MARGINS
%----------------------------------------------------------------------------------------

\usepackage{geometry} % Required for adjusting page dimensions and margins

\geometry{
	left=7.6cm,
	right=1cm,
	top=0.5cm,
	bottom=0.2cm,
	letterpaper, % Paper size, change to letterpaper for US letter size
	% showframe, % Uncomment to show how the type block is set on the page -- typically for debugging
}

% \RequirePackage[left=7.6cm,top=0.1cm,right=1cm,bottom=0.2cm,nohead,nofoot]{geometry}

%----------------------------------------------------------------------------------------
%	COLUMN LAYOUT
%----------------------------------------------------------------------------------------

\usepackage{paracol} % Required for creating multi-column layouts that can span pages automatically

\columnratio{0.55,0.45} % The relative ratios of the two columns in the CV

\setlength\columnsep{0.05\textwidth} % Specify the amount of space between the columns

%----------------------------------------------------------------------------------------
%	FONTS
%----------------------------------------------------------------------------------------
\usepackage{courier}
\usepackage{fontspec} % Required for specifying custom fonts under XeLaTeX

\setmainfont{EBGaramond}[ % Make the default font EBGaramond
Path=fonts/, % The font is provided with the template in the fonts folder
UprightFont=*-Regular.ttf,
BoldFont=*-Bold.ttf,
BoldItalicFont=*-BoldItalic.ttf,
ItalicFont=*-Italic.ttf,
SmallCapsFont=*-SC.ttf
]

\newfontfamily\cvtextfont[Path=fonts/]{freebooterscript} % Create a new font family for the cursive font Freebooter Script, provided with the template in the fonts folder

\newfontfamily\cvlangfont{Courier New} % Create a new font family for programming languages

\newfontfamily{\FA}[Path=fonts/]{FontAwesome} % Create a new font family for FontAwesome icons, provided with the template in the fonts folder
\input{fonts/fontawesomesymbols-xeluatex.tex} % Load a file to create shortcuts to the icons, see icon examples and their shortcuts in fontawesome.pdf in the fonts folder

\usepackage[sf,scale=0.95]{libertine} % Load Libertine as a \sffamily font for sans serif titles

%----------------------------------------------------------------------------------------
%	COLOURS AND LINKS
%----------------------------------------------------------------------------------------

\usepackage[usenames,svgnames]{xcolor} % Allows the definition and use of custom colours

\definecolor{text}{HTML}{2b2b2b} % Main document font colour, off-black
\definecolor{headings}{HTML}{701112} % Dark red colour for headings
\definecolor{shade}{HTML}{F5DD9D} % Peach colour for the contact information box
% \definecolor{linkcolor}{HTML}{641c1d} % 25% desaturated headings colour for links
\definecolor{linkcolor}{HTML}{0000ff} % 25% desaturated headings colour for links
\definecolor{sidecolor}{HTML}{E7E7E7} % Grey sidebar color
\definecolor{python}{HTML}{0E5484}
\definecolor{maingray}{HTML}{B9B9B9}
\definecolor{proglang}{HTML}{f2f2f2}
\definecolor{matlab}{HTML}{e16737}
\definecolor{cuda}{HTML}{3A4E3A}
\definecolor{rpackage}{HTML}{198CE7}
\definecolor{qt}{HTML}{3D6117}
\definecolor{skillcolor}{HTML}{BF656E}

% Other colour options: shade=B9D7D9 and linkcolor=A40000; shade=D4D7FE and linkcolor=FF0080

% For preset colours that can be used by their names without having to define them, see: https://www.latextemplates.com/svgnames-colors

\color{text} % Set the default text colour for the whole document to the colour defined as 'text' above

%------------------------------------------------

\usepackage{hyperref} % Required for links

\hypersetup{colorlinks, breaklinks, urlcolor=linkcolor, linkcolor=linkcolor} % Set up links and their colours

%----------------------------------------------------------------------------------------
%	HEADERS & FOOTERS
%----------------------------------------------------------------------------------------

\usepackage{fancyhdr} % Required for customising headers and footers

\pagestyle{fancy} % Enable custom headers and footers

\fancyhf{} % This suppresses all headers and footers by default, add headers and footers in the template file as per the example

\renewcommand{\headrulewidth}{0pt} % Remove the default rule under the header

%----------------------------------------------------------------------------------------
%	SECTIONS
%----------------------------------------------------------------------------------------

\usepackage[nobottomtitles*]{titlesec} % Required for modifying sections, the nobottomtitles* is required for pushing section titles to the next page when they are close to the bottom of the page

\renewcommand{\bottomtitlespace}{0.1\textheight} % Modify the minimal space required from the bottom margin not to move the title to the next page

\titleformat{\section}{\color{headings}\scshape\LARGE\raggedright}{}{0em}{}[\color{black}\titlerule] % Define the \section format

\titlespacing{\section}{0pt}{0pt}{8pt} % Spacing around section titles, the order is: left, before and after

%----------------------------------------------------------------------------------------
%	CUSTOM COMMANDS
%----------------------------------------------------------------------------------------

% Command for entering a new work position
\newcommand{\workposition}[5]{
	{\expandafter\ifstrequal\expandafter{#3}{}{}{{\raggedright\large #3}}} % Employer
	\hfill
	{\raggedleft\textsc{#1\expandafter\ifstrequal\expandafter{#2}{}{}{\hspace{2pt}\footnotesize{(#2)}}}\par} % Duration and conditional full time/part time text
	\expandafter\ifstrequal\expandafter{#4}{}{}{{\raggedright\large\textit{\textbf{#4}}}\\[4pt]} % Job title
	\expandafter\ifstrequal\expandafter{#5}{}{}{#5\\} % Description
}

% Command for entering a separate qualification
\newcommand{\educationentry}[5]{
	\textsc{#1} & \textbf{#2}\\ % Duration and degree
	\expandafter\ifstrequal\expandafter{#3}{}{}{& {\small\textsc{#3}}\\} % Honours, achievements or distinctions (e.g. first class honours)
	\expandafter\ifstrequal\expandafter{#4}{}{}{& #4\\} % Department
	\expandafter\ifstrequal\expandafter{#5}{}{}{& \textit{#5}\\[6pt]} % Institution
}

% Command for entering a separate table row -- used as a generic visual element for any section that uses a two column table
\newcommand{\tableentry}[3]{
	\textsc{#1} & #2\expandafter\ifstrequal\expandafter{#3}{}{\\}{\\[6pt]} % First the heading, then content, then a conditional insertion of whitespace if the third parameter has any content in it
}

% Command for entering a separate table row -- used as a generic visual element for any section that uses a three column table
\newcommand{\tableentrythree}[4]{
	\textsc{#1} & {#2} & {#3}\expandafter\ifstrequal\expandafter{#4}{}{\\}{\\[6pt]} % First the heading, then content, then the location, then a conditional insertion of whitespace if the third parameter has any content in it
}

% Command for entering a long-form description where there is a title on one line and a paragraph description below it
\newcommand{\longformdescription}[2]{
	\textit{#1}\\[3pt]
	#2\medskip
}

% Command for entering a publication in long-form format
\newcommand{\longformpublication}[1]{
	#1\medskip
}

% Command for entering a publication as a short DOI (digital object identifier) string to the publication, a link is automatically created
\newcommand{\doipublication}[5]{
	#1 & % Year
	\href{#2}{\expandafter\ifstrequal\expandafter{#3}{firstauthor}{\small#5}{\small#5}}% title string and if "firstauthor" is entered for the 3rd argument, this makes the DOI string bold indicating a first author publication
	\expandafter\ifstrequal\expandafter{#4}{}{\\}{\\[3pt]} % Conditional insertion of whitespace if the 4th parameter has any content in it
}

% Command to create the rounded boxes around the first three letters of section titles
\newcommand*\round[2]{%
	\begin{tikzpicture}[baseline=(char.base)]
		\node[align=center,anchor=north west, draw,rectangle, rounded corners=2.5mm, inner sep=1.6pt, minimum size=5.5mm, text centered, minimum height = 5mm, text height=2mm, fill=#2,#2,text=black](char){#1};%
	\end{tikzpicture}
}

% Command for creating a big colored dot
\newcommand{\programlang}[2]{
	\round{
		{#1$\bullet$}~~{\footnotesize\cvlangfont#2}
	}{proglang}
}

% Command for creating skill level plots
\newcommand{\skilllevel}[2]{%
    \textcolor{black}{\textsc{#1}}~~~~~~\hfill
    \foreach \x in {1,...,7}{%
      \space{\ifnumgreater{\x}{#2}{\color{maingray}\faCircleO}{\color{skillcolor}\faDotCircleO}}}\par%
}

%----------------------------------------------------------------------------------------
%	GRAPHICS DEFINITIONS
%---------------------------------------------------------------------------------------- 

\usepackage{tikz} % Required for creating the plots
\usetikzlibrary{shapes, backgrounds}
\tikzset{x=1cm, y=1cm} % Default tikz units

% Command to vertically centre adjacent content
\newcommand{\vcenteredhbox}[1]{% The only parameter is for the content to centre
	\begingroup%
		\setbox0=\hbox{#1}\parbox{\wd0}{\box0}%
	\endgroup%
}

%----------------------------------------------------------------------------------------
%	CHARTS
%---------------------------------------------------------------------------------------- 

\newcounter{barcount}

% Environment to hold a new bar chart
\newenvironment{barchart}[1]{ % The only parameter is the maximum bar width, in cm
	\newcommand{\barwidth}{0.35}
	\newcommand{\barsep}{0.2}
	
	% Command to add a bar to the bar chart
	\newcommand{\baritem}[2]{ % The first argument is the bar label and the second is the percentage the current bar should take up of the total width
		\pgfmathparse{##2}
		\let\perc\pgfmathresult
		
		\pgfmathparse{#1}
		\let\barsize\pgfmathresult
		
		\pgfmathparse{\barsize*##2/100}
		\let\barone\pgfmathresult
		
		\pgfmathparse{(\barwidth*\thebarcount)+(\barsep*\thebarcount)}
		\let\barx\pgfmathresult
		
		\filldraw[fill=shade, draw=none] (0,-\barx) rectangle (\barone,-\barx-\barwidth);
		
		\node [label=180:\colorbox{white}{\textcolor{text}{##1}}] at (0,-\barx-0.175) {};
		\addtocounter{barcount}{1}
	}
	\begin{tikzpicture}
		\setcounter{barcount}{0}
}{
	\end{tikzpicture}
}

%----------------------------------------------------------------------------------------
%	 SIDEBAR DEFINITIONS
%----------------------------------------------------------------------------------------

\usepackage[absolute,overlay]{textpos}
\usepackage{ifthen}

\newlength\imagewidth
\newlength\imagescale
\pgfmathsetlength{\imagewidth}{5cm}
\pgfmathsetlength{\imagescale}{\imagewidth/600}

\newlength{\TotalSectionLength} % Define a new length to hold the remaining line width after the section title is printed
\newlength{\SectionTitleLength} % Define a new length to hold the width of the section title
\newcommand{\profilesection}[1]{%
	\setlength\TotalSectionLength{\linewidth}% Set the total line width
	\settowidth{\SectionTitleLength}{\huge #1 }% Calculate the width of the section title
	% \addtolength\TotalSectionLength{-\SectionTitleLength}% Subtract the section title width from the total width
	\addtolength\TotalSectionLength{-2.22221pt}% Modifier to remove overfull box warning
	\vspace{5pt}% Whitespace before the section title
	{\color{headings} \scshape\LARGE #1 \\\rule[0.3\baselineskip]{\TotalSectionLength}{0.1pt}}% Print the title and auto-width rule
}

% \titleformat{\section}{\color{headings}\scshape\LARGE\raggedright}{}{0em}{}[\color{black}\titlerule] % Define the \section format


% Define custom commands for CV info
\newcommand{\cvdate}[1]{\renewcommand{\cvdate}{#1}}
\newcommand{\cvmail}[1]{\renewcommand{\cvmail}{#1}}
\newcommand{\cvnumberphone}[1]{\renewcommand{\cvnumberphone}{#1}}
\newcommand{\cvaddress}[1]{\renewcommand{\cvaddress}{#1}}
\newcommand{\cvsite}[1]{\renewcommand{\cvsite}{#1}}
\newcommand{\aboutme}[1]{\renewcommand{\aboutme}{#1}}
\newcommand{\education}[1]{\renewcommand{\education}{#1}}
\newcommand{\references}[1]{\renewcommand{\references}{#1}}
\newcommand{\additional}[1]{\renewcommand{\additional}{#1}}
\newcommand{\publications}[1]{\renewcommand{\publications}{#1}}
\newcommand{\awards}[1]{\renewcommand{\awards}{#1}}
\newcommand{\profilepic}[1]{\renewcommand{\profilepic}{#1}}
\newcommand{\cvname}[1]{\renewcommand{\cvname}{#1}}
\newcommand{\cvjobtitle}[1]{\renewcommand{\cvjobtitle}{#1}}
\newcommand{\cvgithub}[1]{\renewcommand{\cvgithub}{#1}}
\newcommand{\cvlinkedin}[1]{\renewcommand{\cvlinkedin}{#1}}

% Command for printing skill progress bars
\newcommand\skills[1]{ 
	\renewcommand{\skills}{
		\begin{tikzpicture}
			\foreach [count=\i] \x/\y in {#1}{
				\draw[fill=maingray,maingray] (0,\i) rectangle (6,\i+0.4);
				\draw[fill=white,skillcolor](0,\i) rectangle (\y,\i+0.4);
				\node [above right] at (0,\i+0.4) {\x};
			}
		\end{tikzpicture}
	}
}

%----------------------------------------------------------------------------------------
%	 SIDEBAR LAYOUT
%----------------------------------------------------------------------------------------
\usepackage{ragged2e} % For fully justified text

\newcommand{\makeprofile}{
	\begin{tikzpicture}[remember picture,overlay]
		\node [rectangle, fill=sidecolor, anchor=north, minimum width=9cm, minimum height=\paperheight+1cm] (box) at (-5cm,2.5cm){};
	\end{tikzpicture}

	%------------------------------------------------

	\begin{textblock}{6}(0.5, 0.2)

		%------------------------------------------------
		
		\ifthenelse{\equal{\profilepic}{}}{}{
			\begin{center}
				\begin{tikzpicture}[x=\imagescale,y=-\imagescale]
					\clip (600/2, 567/2) circle (567/2);
					\node[anchor=north west, inner sep=0pt, outer sep=0pt] at (0,0) {\includegraphics[width=\imagewidth]{\profilepic}};
				\end{tikzpicture}
			\end{center}
		}

		% %------------------------------------------------

		% {\Huge\color{mainblue}\cvname}

		% %------------------------------------------------

		% {\Large\color{black!80}\cvjobtitle}

		%------------------------------------------------

		% \renewcommand{\arraystretch}{1.2}
		% \begin{tabular}{p{0.5cm} @{\hskip 0.5cm}p{5cm}}
		% 	\ifthenelse{\equal{\cvdate}{}}{}{\raisebox{-1pt}{\faHome} & \cvdate\\}
		% 	\ifthenelse{\equal{\cvaddress}{}}{}{\raisebox{-1pt}{\faHome} & \cvaddress \\} % Address
		% 	\ifthenelse{\equal{\cvnumberphone}{}}{}{\raisebox{-1pt}{\faPhone} & \cvnumberphone\\}
		% 	\ifthenelse{\equal{\cvsite}{}}{}{\raisebox{-1pt}{\faGithub} & \cvsite\\}
		% 	\ifthenelse{\equal{\cvmail}{}}{}{\raisebox{0pt}{\small\faEnvelope} & \href{mailto:\cvmail}{\cvmail}}
		% \end{tabular}

		\parbox[top][0.12\textheight][c]{\linewidth}{ % Parbox to hold the colour box; fixed height to match the name/CV text to the left, centred vertically and full line width
			\vspace{0\textheight} % Reduce whitespace above the parbox to separate it from the main content
			\colorbox{shade}{ % Create the coloured box
				\begin{supertabular}{p{0.05\linewidth}|p{0.775\linewidth}} % Start a table with two columns, the table will ensure everything is aligned
					\ifthenelse{\equal{\cvdate}{}}{}{\raisebox{-1pt}{\faInfo} & \cvdate \\}
					\ifthenelse{\equal{\cvaddress}{}}{}{\raisebox{-1pt}{\faHome} & \cvaddress \\} % Address
					\ifthenelse{\equal{\cvaddress}{}}{}{\raisebox{-1pt}{\faPhone} & \cvnumberphone \\} % Phone number
					\ifthenelse{\equal{\cvmail}{}}{}{\raisebox{0pt}{\small\faEnvelope} & \href{mailto:\cvmail}{\cvmail} \\} % Email address
					\ifthenelse{\equal{\cvsite}{}}{}{\raisebox{-1pt}{\small\faDesktop} & \href{https://\cvsite}{\cvsite} \\} % Website
					\ifthenelse{\equal{\cvgithub}{}}{}{\raisebox{-1pt}{\faGithub} & \href{https://\cvgithub}{\cvgithub} \\} % GitHub profile
					\ifthenelse{\equal{\cvlinkedin}{}}{}{\raisebox{-1pt}{\faLinkedinSquare} & \href{https://\cvlinkedin}{\cvlinkedin} \\} % LinkedIn profile
					% See fontawesome.pdf in the fonts folder for all icons you can use
				\end{supertabular}
			}
		}

		%------------------------------------------------

		\ifthenelse{\equal{\aboutme}{}}{}{
			\profilesection{About me}
			\vspace{-\baselineskip}
			\begin{justify}
				\aboutme
			\end{justify}
		}

		%------------------------------------------------

		\ifthenelse{\equal{\education}{}}{}{
			\profilesection{Education}\\
			\education
		}

		%------------------------------------------------

		\profilesection{Expertise}\\
		\skills
		
		%------------------------------------------------

		\ifthenelse{\equal{\additional}{}}{}{
			\profilesection{Research interests}\\
			\additional
		}

		%------------------------------------------------
		
		\ifthenelse{\equal{\awards}{}}{}{
			\profilesection{Awards}
			\awards
		}
		
		%------------------------------------------------
		
		\ifthenelse{\equal{\references}{}}{}{
			\profilesection{References}\\
			\references
		}

		%------------------------------------------------

		\ifthenelse{\equal{\publications}{}}{}{
			\profilesection{Publications}\\
			\publications
		}


	\end{textblock}
}