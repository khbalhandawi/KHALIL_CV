%%%%%%%%%%%%%%%%%%%%%%%%%%%%%%%%%%%%%%%%%
% Freeman Curriculum Vitae
% XeLaTeX Template
% Version 2.0 (19/3/2018)
%
% This template originates from:
% http://www.LaTeXTemplates.com
%
% Authors:
% Vel (vel@LaTeXTemplates.com)
% Alessandro Plasmati
%
% License:
% CC BY-NC-SA 3.0 (http://creativecommons.org/licenses/by-nc-sa/3.0/)
%
%!TEX program = xelatex
% NOTICE: This template must be compiled with XeLaTeX, the line above should
% ensure this happens automatically but if it doesn't you will need to specify 
% XeLaTeX as the engine in your editor or script
% 
%%%%%%%%%%%%%%%%%%%%%%%%%%%%%%%%%%%%%%%%%

%----------------------------------------------------------------------------------------
%	PACKAGES AND OTHER DOCUMENT CONFIGURATIONS
%----------------------------------------------------------------------------------------

\documentclass[12pt]{article} % Font size, can be: 10pt, 11pt or 12pt

%%%%%%%%%%%%%%%%%%%%%%%%%%%%%%%%%%%%%%%%%
% Freeman Curriculum Vitae
% Structure Specification File
% Version 1.0 (19/3/2018)
%
% This template originates from:
% http://www.LaTeXTemplates.com
%
% Authors:
% Vel (vel@LaTeXTemplates.com)
% Alessandro Plasmati
%
% License:
% CC BY-NC-SA 3.0 (http://creativecommons.org/licenses/by-nc-sa/3.0/)
% 
%%%%%%%%%%%%%%%%%%%%%%%%%%%%%%%%%%%%%%%%%

%----------------------------------------------------------------------------------------
%	PACKAGES AND OTHER DOCUMENT CONFIGURATIONS
%----------------------------------------------------------------------------------------

\usepackage{etoolbox} % Required for conditional statements

\setlength\parindent{0pt} % Stop paragraph indentation

\usepackage{supertabular} % Required for the supertabular environment which allows tables to span multiple pages so sections with tables correctly wrap across pages

\usepackage{array} % for custom alignment if supertabular columns

\usepackage{ifthen} % For creating conditional entries

\usepackage{enumitem} % for formatting bulleted lists and reducing whitespace before bullet list

%----------------------------------------------------------------------------------------
%	DOCUMENT MARGINS
%----------------------------------------------------------------------------------------

\usepackage{geometry} % Required for adjusting page dimensions and margins

\geometry{
	hmargin=1.5cm, % Horizontal margin
	vmargin=1.75cm, % Vertical margin
	letterpaper, % Paper size, change to letterpaper for US letter size
	%showframe, % Uncomment to show how the type block is set on the page -- typically for debugging
}

%----------------------------------------------------------------------------------------
%	COLUMN LAYOUT
%----------------------------------------------------------------------------------------

\usepackage{paracol} % Required for creating multi-column layouts that can span pages automatically

\columnratio{0.55,0.45} % The relative ratios of the two columns in the CV

\setlength\columnsep{0.05\textwidth} % Specify the amount of space between the columns

%----------------------------------------------------------------------------------------
%	FONTS
%----------------------------------------------------------------------------------------

\usepackage{fontspec} % Required for specifying custom fonts under XeLaTeX

\setmainfont{EBGaramond}[ % Make the default font EBGaramond
Path=fonts/, % The font is provided with the template in the fonts folder
UprightFont=*-Regular.ttf,
BoldFont=*-Bold.ttf,
BoldItalicFont=*-BoldItalic.ttf,
ItalicFont=*-Italic.ttf,
SmallCapsFont=*-SC.ttf
]

\newfontfamily\cvtextfont[Path=fonts/]{freebooterscript} % Create a new font family for the cursive font Freebooter Script, provided with the template in the fonts folder

\newfontfamily\cvlangfont{Courier New} % Create a new font family for programming languages

\newfontfamily{\FA}[Path=fonts/]{FontAwesome} % Create a new font family for FontAwesome icons, provided with the template in the fonts folder
\input{fonts/fontawesomesymbols-xeluatex.tex} % Load a file to create shortcuts to the icons, see icon examples and their shortcuts in fontawesome.pdf in the fonts folder

\usepackage[sf,scale=0.95]{libertine} % Load Libertine as a \sffamily font for sans serif titles

%----------------------------------------------------------------------------------------
%	COLOURS AND LINKS
%----------------------------------------------------------------------------------------

\usepackage[usenames,svgnames]{xcolor} % Allows the definition and use of custom colours

\definecolor{text}{HTML}{2b2b2b} % Main document font colour, off-black
\definecolor{headings}{HTML}{701112} % Dark red colour for headings
\definecolor{shade}{HTML}{F5DD9D} % Peach colour for the contact information box
\definecolor{linkcolor}{HTML}{0000ff} % 25% desaturated headings colour for links
\definecolor{python}{HTML}{0E5484}
\definecolor{maingray}{HTML}{B9B9B9}
\definecolor{proglang}{HTML}{f2f2f2}
\definecolor{matlab}{HTML}{e16737}
\definecolor{cuda}{HTML}{3A4E3A}
\definecolor{rpackage}{HTML}{198CE7}
\definecolor{qt}{HTML}{3D6117}
% Other colour options: shade=B9D7D9 and linkcolor=A40000; shade=D4D7FE and linkcolor=FF0080

% For preset colours that can be used by their names without having to define them, see: https://www.latextemplates.com/svgnames-colors

\color{text} % Set the default text colour for the whole document to the colour defined as 'text' above

%------------------------------------------------

\usepackage{hyperref} % Required for links

\hypersetup{colorlinks, breaklinks, urlcolor=linkcolor, linkcolor=linkcolor} % Set up links and their colours


%----------------------------------------------------------------------------------------
%	HEADERS & FOOTERS
%----------------------------------------------------------------------------------------

\usepackage{fancyhdr} % Required for customising headers and footers

\pagestyle{fancy} % Enable custom headers and footers

\fancyhf{} % This suppresses all headers and footers by default, add headers and footers in the template file as per the example

\renewcommand{\headrulewidth}{0pt} % Remove the default rule under the header

%----------------------------------------------------------------------------------------
%	SECTIONS
%----------------------------------------------------------------------------------------

\usepackage[nobottomtitles*]{titlesec} % Required for modifying sections, the nobottomtitles* is required for pushing section titles to the next page when they are close to the bottom of the page

\renewcommand{\bottomtitlespace}{0.1\textheight} % Modify the minimal space required from the bottom margin not to move the title to the next page

\titleformat{\section}{\color{headings}\scshape\LARGE\raggedright}{}{0em}{}[\color{black}\titlerule] % Define the \section format

\titlespacing{\section}{0pt}{0pt}{8pt} % Spacing around section titles, the order is: left, before and after

%----------------------------------------------------------------------------------------
%	GRAPHICS DEFINITIONS
%---------------------------------------------------------------------------------------- 

\usepackage{tikz} % Required for creating the plots
\usetikzlibrary{shapes, backgrounds}
\tikzset{x=1cm, y=1cm} % Default tikz units

% Command to vertically centre adjacent content
\newcommand{\vcenteredhbox}[1]{% The only parameter is for the content to centre
	\begingroup%
		\setbox0=\hbox{#1}\parbox{\wd0}{\box0}%
	\endgroup%
}

%----------------------------------------------------------------------------------------
%	CHARTS
%---------------------------------------------------------------------------------------- 

\newcounter{barcount}

% Environment to hold a new bar chart
\newenvironment{barchart}[1]{ % The only parameter is the maximum bar width, in cm
	\newcommand{\barwidth}{0.35}
	\newcommand{\barsep}{0.2}
	
	% Command to add a bar to the bar chart
	\newcommand{\baritem}[2]{ % The first argument is the bar label and the second is the percentage the current bar should take up of the total width
		\pgfmathparse{##2}
		\let\perc\pgfmathresult
		
		\pgfmathparse{#1}
		\let\barsize\pgfmathresult
		
		\pgfmathparse{\barsize*##2/100}
		\let\barone\pgfmathresult
		
		\pgfmathparse{(\barwidth*\thebarcount)+(\barsep*\thebarcount)}
		\let\barx\pgfmathresult
		
		\filldraw[fill=shade, draw=none] (0,-\barx) rectangle (\barone,-\barx-\barwidth);
		
		\node [label=180:\colorbox{white}{\textcolor{text}{##1}}] at (0,-\barx-0.175) {};
		\addtocounter{barcount}{1}
	}
	\begin{tikzpicture}
		\setcounter{barcount}{0}
}{
	\end{tikzpicture}
}

%----------------------------------------------------------------------------------------
%	CUSTOM COMMANDS
%----------------------------------------------------------------------------------------

% Command for entering a new work position
\newcommand{\workposition}[7]{
	\multicolumn{2}{c}{
		\expandafter\ifstrequal\expandafter{#3}{}{}{\textbf{#3}} % Employer
		\expandafter\ifstrequal\expandafter{#4}{}{}{\hfill {\raggedright\textsc{#4}}} % Location
	}
	\expandafter\ifstrequal\expandafter{#3#4}{}{\\[-10pt]}{\\}
	\textsc{
		\expandafter\ifstrequal\expandafter{#1}{}{}{#1}
		\expandafter\ifstrequal\expandafter{#2}{}{}{\hspace{6pt}\footnotesize{(#2)}}
	} % Duration and conditional full time/part time text
	\expandafter\ifstrequal\expandafter{#5}{}{}{& {\textit{{#5}}}\\[4pt]} % Job title
	\expandafter\ifstrequal\expandafter{#6}{}{}{& #6} % Description
	\expandafter\ifstrequal\expandafter{#7}{}{}{\\[6pt]}
}

% Command for entering a separate qualification
\newcommand{\educationentry}[5]{
	\textsc{#1} & \textbf{#2} % Duration and degree
	\expandafter\ifstrequal\expandafter{#5}{}{}{& {\raggedright\textit{#5}}\\} % Institution
	\expandafter\ifstrequal\expandafter{#4}{}{}{& #4} % Department
	\expandafter\ifstrequal\expandafter{#3}{}{}{, {\small\textsc{#3}}\\} % Honours, achievements or distinctions (e.g. first class honours)
}

% Command for entering a separate table row -- used as a generic visual element for any section that uses a two column table
\newcommand{\tableentry}[3]{
	\textsc{#1} % for bullets
	& #2 % main text
	\expandafter\ifstrequal\expandafter{#3}{}{\\}{\\[6pt]} % First the heading, then content, then a conditional insertion of whitespace if the third parameter has any content in it
}

% Command for entering a separate table row -- used as a generic visual element for any section that uses a two column table
\newcommand{\tableentryraw}[3]{
	#1 & #2\expandafter\ifstrequal\expandafter{#3}{}{\\}{\\[6pt]} % First the heading, then content, then a conditional insertion of whitespace if the third parameter has any content in it
}

% Command for entering a separate table row -- used as a generic visual element for any section that uses a two column table
\newcommand{\tableentrythree}[4]{
	\textsc{#1} & #2 & \textsc{#3}\expandafter\ifstrequal\expandafter{#4}{}{\\}{\\[6pt]} % First the heading, then content, then the location, then a conditional insertion of whitespace if the third parameter has any content in it
}

% Command for entering a separate table row -- used as a generic visual element for any section that uses a two column table
\newcommand{\tableentryrawthree}[4]{
	\textsc{#1} & #2 & #3\expandafter\ifstrequal\expandafter{#4}{}{\\}{\\[6pt]} % First the heading, then content, then the location, then a conditional insertion of whitespace if the third parameter has any content in it
}


% Command for entering a long-form description where there is a title on one line and a paragraph description below it
\newcommand{\longformdescription}[2]{
	\textit{#1}\\[3pt]
	#2\medskip
}

% Command for entering a long-form description where there is a title on one line and a paragraph description below it
\newcommand{\longformdescriptiontight}[2]{
	\textit{#1}
	#2\medskip
}

% Command for entering a publication in long-form format
\newcommand{\longformpublication}[1]{
	#1\medskip
}

% Command for entering a publication as a short DOI (digital object identifier) string to the publication, a link is automatically created
\newcommand{\doipublication}[4]{
	#1 & % Year
	\href{http://dx.doi.org/#2}{\expandafter\ifstrequal\expandafter{#3}{firstauthor}{\textbf{doi:#2}}{doi:#2}}% DOI string and if "firstauthor" is entered for the 3rd argument, this makes the DOI string bold indicating a first author publication
	\expandafter\ifstrequal\expandafter{#4}{}{\\}{\\[3pt]} % Conditional insertion of whitespace if the 4th parameter has any content in it
}

% Command for creating skill level plots
\newcommand{\skilllevel}[2]{%
    \textcolor{black}{\textbf{#1}}~~~~~~\hfill
    \foreach \x in {1,...,10}{%
      \space{\ifnumgreater{\x}{#2}{\color{maingray}\faCircleO}{\color{headings}\faDotCircleO}}}\par%
}

% Define custom commands for CV info
\newcommand{\cvdate}[1]{\renewcommand{\cvdate}{#1}}
\newcommand{\cvmail}[1]{\renewcommand{\cvmail}{#1}}
\newcommand{\cvnumberphone}[1]{\renewcommand{\cvnumberphone}{#1}}
\newcommand{\cvaddress}[1]{\renewcommand{\cvaddress}{#1}}
\newcommand{\cvsite}[1]{\renewcommand{\cvsite}{#1}}
\newcommand{\aboutme}[1]{\renewcommand{\aboutme}{#1}}
\newcommand{\education}[1]{\renewcommand{\education}{#1}}
\newcommand{\references}[1]{\renewcommand{\references}{#1}}
\newcommand{\awards}[1]{\renewcommand{\awards}{#1}}
\newcommand{\profilepic}[1]{\renewcommand{\profilepic}{#1}}
\newcommand{\cvname}[1]{\renewcommand{\cvname}{#1}}
\newcommand{\cvjobtitle}[1]{\renewcommand{\cvjobtitle}{#1}}
\newcommand{\cvgithub}[1]{\renewcommand{\cvgithub}{#1}}
\newcommand{\cvlinkedin}[1]{\renewcommand{\cvlinkedin}{#1}}
\newcommand{\cvdescribe}[1]{\renewcommand{\cvdescribe}{#1}}
\newcommand{\cvskills}[1]{\renewcommand{\cvskills}{#1}}
\newcommand{\cvhobbies}[1]{\renewcommand{\cvhobbies}{#1}}
\newcommand{\cvreferences}[1]{\renewcommand{\cvreferences}{#1}}
\newcommand{\cvcourses}[1]{\renewcommand{\cvcourses}{#1}}

% Command to create the rounded boxes around the first three letters of section titles
\newcommand*\round[2]{%
	\begin{tikzpicture}[baseline=(char.base)]
		\node[align=center,anchor=north west, draw,rectangle, rounded corners=2.5mm, inner sep=1.6pt, minimum size=5.5mm, text centered, minimum height = 5mm, text height=2mm, fill=#2,#2,text=black](char){#1};%
	\end{tikzpicture}
}

% Command for creating a big colored dot
\newcommand{\programlang}[2]{
	\round{
		{#1$\bullet$}~~{\footnotesize\cvlangfont#2}
	}{proglang}
} % Include the file that specifies the document structure

% Headers and footers can be added with the \lhead{} \rhead{} \lfoot{} \rfoot{} commands
% Example right footer:
%\rfoot{\color{headings}{\sffamily Last update: \today. Typeset with Xe\LaTeX}}

%----------------------------------------------------------------------------------------
%	 PERSONAL INFORMATION
%----------------------------------------------------------------------------------------

% If you don't need one or more of the below, just remove the content leaving the command, e.g. \cvnumberphone{}

\profilepic{} % Profile picture

\cvname{Khalil Al Handawi, PhD} % Your name
\cvjobtitle{Engineer, designer, and researcher} % Job title/career

\cvdate{} % Date of birth
\cvaddress{Montr\'{e}al Qu\'{e}bec, Canada} % Short address/location, use \newline if more than 1 line is required
\cvnumberphone{+1 (514) 572-7367} % Phone number
\cvsite{khbalhandawi.github.io} % Personal website
\cvmail{khalil.alhandawi@mail.mcgill.ca} % Email address
\cvgithub{github.com/khbalhandawi} % GitHub
\cvlinkedin{linkedin.com/in/khbalhandawi} % LinkedIn

\aboutme{I believe that physics and artificial intelligence should be two sides of the same coin. One cannot exist without the other. How? By cross-validation. In this way, the toughest physics and mathematics problems can be solved! This philosophy is what drives my research.} % To have no About Me section, just remove all the text and leave \aboutme{}

\begin{document}

%----------------------------------------------------------------------------------------
%	ADDRESS
%----------------------------------------------------------------------------------------

Khalil Al Handawi, PhD\\
500 Avenue Des Pins Ouest\\
H2W 1S7, Montr\'{e}al, Qu\'{e}bec, Canada\\
\faPhone~~\cvnumberphone\\
\faEnvelope~~\href{mailto:\cvmail}{\cvmail}\\

\today\\

Camera hardware engineering team\\
Santa Clara Valley (Cupertino), CA, United States\\
Subject: Camera Mechanical Engineer - FEA\\
Role Number: 200483677\\[6pt] \medskip

%----------------------------------------------------------------------------------------
%	MAIN BODY
%----------------------------------------------------------------------------------------

Dear Hiring Manager,

\medskip % Extra whitespace before the next section 

I am writing to express my interest in the Camera Mechanical Engineer position within the camera hardware engineering team at Apple, Inc. I am excited about the opportunity to leverage my optimization, simulation, and modeling experience to enable the innovations behind Apple's products.

\medskip % Extra whitespace before the next section 

I understand that as part of the role, I will be developing simulation and modeling solutions pertinent to camera design using commercial finite element software. I have extensive experience with Abaqus and NX Siemens from my research at McGill University, Chalmers University, and collaboration with GKN Aerospace. I have 6 years of experience within aviation and the aerospace industry at said institutions, where thermal modeling and structural simulation are an integral part of my work.

\medskip % Extra whitespace before the next section 

During my doctoral and postdoctoral research, I modeled additive manufacturing (AM) processes on aeroengine components using thermomechanical modeling to capture the residual stress and deformation due to the localized temperature gradients caused by a moving heat source. The entire simulation and design workflow was automated using API scripting in various commercial software to generate and analyze parametric CAD models (generated using NX Seimens and analyzed using Abaqus CAE's coupled structural and thermal solver) which gave me the necessary exposure to simulation tools that are commonly used in the industry. I also used advanced functionality within Abaqus such as user-defined subroutines (UMAT and VUMAT) to define material properties following thermal deposition. 

\medskip % Extra whitespace before the next section 

I also solved robust design optimization problems where some of the design requirements are modeled by probabilistic functions. I authored a \href{https://sed-group.github.io/mvmlib/index.html}{Python library} and a \href{https://github.com/khbalhandawi/scale_AM_webapp}{web application} to support the design activities of our industry partner, GKN Aerospace engine systems.

\medskip % Extra whitespace before the next section 

In addition to my simulation skills, I have authored several software and simulation programs for research purposes in various application domains (such as healthcare and commercial aviation). As an example, I wrote a \href{https://github.com/khbalhandawi/COVID_SIM_GPU}{CUDA accelerated C++ simulation} for the simulation of pandemics giving me exposure to high performance computing and multithreading. I was also an adjunct lecturer at McGill University, teaching the engineering systems optimization course (MECH559) to engineering students. I authored several \href{https://github.com/khbalhandawi/MECH559_notebooks}{notebooks in Python and Julia} to help the students understand the implementation of said algorithms and solve real-world engineering problems in their projects. This skillset could allow me to develop in-house simulation capabilities that could potentially address the limitations of commercial solvers while also providing relevant training and workshops on the use of said tools.

\medskip % Extra whitespace before the next section 

I am currently a post-doctoral researcher at the department of computer science and operations research (DIRO) at the Universit\'{e} de Montr\'{e}al as part of an industrial project with the international air transport association (IATA). My current research focuses on graph representation learning from aviation data collected over the last decade to assess the effectiveness and impact of the IATA operation safety audit (IOSA) on air travel accessibility and cooperation between airlines. I am specifically focusing on unsupervised learning on graph data structures to identify community structures within networks.

\medskip % Extra whitespace before the next section 

I believe these experiences are relevant to the role in the following ways: 

\begin{itemize}
	\item I can provide structural and thermal simulation solutions for Apple's camera products since I can develop and validate models (both simulation-based, and statistical) given my previous modeling and simulation experience during aerospace design, healthcare, and aviation related projects. 
	\item I can interpret finite element results and models using my advanced background in solid mechanics (through my Ph.D. courses: \href{https://www.mcgill.ca/study/2022-2023/courses/mech-632}{MECH632}).
	\item I can navigate commercial CAD and FEA modeling software given my experience with static and transient thermal simulations.
	\item I can automate our design processes using my knowledge of parameteric CAD and FEA modeling and use of their APIs and subroutines.
	\item I can help develop in-house simulation capabilities using my knowledge of HPC and low-level programming languages such as C++ and CUDA.
	\item I can support design efforts to accommodate moving performance targets and requirements through my extensive knowledge on robust design and design for changeability from my doctoral research.
	\item I can apply software development best-practices such as unit-testing, documentation, and maintenance given my experience in authoring a Python library for use by other researchers and engineers. This will allow me to communicate our solutions with other teams.
	\item I have a good understanding of the industry's simulation needs given my experience working with various industry partners such as GKN Aerospace.
	\item I can combine my domain knowledge in mechanical engineering and physics-based modeling with software engineering given the multidisciplinary nature of my research.
\end{itemize}

\medskip % Extra whitespace before the next section 

Thank you for considering my application. I would be honored to have the opportunity to discuss my qualifications further and show you my \href{https://khbalhandawi.github.io/projects/}{portfolio} of projects. Please feel free to contact me through any of the channels at the top of this letter.

\medskip % Extra whitespace before the next section

Best regards,

\medskip % Extra whitespace before the next section


Khalil Al Handawi

% \begin{figure*}[h]
% 	\includegraphics[width=0.2\textwidth]{Signiture.png}
% \end{figure*}

\medskip % Extra whitespace before the next section

%----------------------------------------------------------------------------------------

\end{document}