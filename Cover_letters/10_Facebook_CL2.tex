%%%%%%%%%%%%%%%%%%%%%%%%%%%%%%%%%%%%%%%%%
% Freeman Curriculum Vitae
% XeLaTeX Template
% Version 2.0 (19/3/2018)
%
% This template originates from:
% http://www.LaTeXTemplates.com
%
% Authors:
% Vel (vel@LaTeXTemplates.com)
% Alessandro Plasmati
%
% License:
% CC BY-NC-SA 3.0 (http://creativecommons.org/licenses/by-nc-sa/3.0/)
%
%!TEX program = xelatex
% NOTICE: This template must be compiled with XeLaTeX, the line above should
% ensure this happens automatically but if it doesn't you will need to specify 
% XeLaTeX as the engine in your editor or script
% 
%%%%%%%%%%%%%%%%%%%%%%%%%%%%%%%%%%%%%%%%%

%----------------------------------------------------------------------------------------
%	PACKAGES AND OTHER DOCUMENT CONFIGURATIONS
%----------------------------------------------------------------------------------------

\documentclass[12pt]{article} % Font size, can be: 10pt, 11pt or 12pt

%%%%%%%%%%%%%%%%%%%%%%%%%%%%%%%%%%%%%%%%%
% Freeman Curriculum Vitae
% Structure Specification File
% Version 1.0 (19/3/2018)
%
% This template originates from:
% http://www.LaTeXTemplates.com
%
% Authors:
% Vel (vel@LaTeXTemplates.com)
% Alessandro Plasmati
%
% License:
% CC BY-NC-SA 3.0 (http://creativecommons.org/licenses/by-nc-sa/3.0/)
% 
%%%%%%%%%%%%%%%%%%%%%%%%%%%%%%%%%%%%%%%%%

%----------------------------------------------------------------------------------------
%	PACKAGES AND OTHER DOCUMENT CONFIGURATIONS
%----------------------------------------------------------------------------------------

\usepackage{etoolbox} % Required for conditional statements

\setlength\parindent{0pt} % Stop paragraph indentation

\usepackage{supertabular} % Required for the supertabular environment which allows tables to span multiple pages so sections with tables correctly wrap across pages

\usepackage{array} % for custom alignment if supertabular columns

\usepackage{ifthen} % For creating conditional entries

\usepackage{enumitem} % for formatting bulleted lists and reducing whitespace before bullet list

%----------------------------------------------------------------------------------------
%	DOCUMENT MARGINS
%----------------------------------------------------------------------------------------

\usepackage{geometry} % Required for adjusting page dimensions and margins

\geometry{
	hmargin=1.5cm, % Horizontal margin
	vmargin=1.75cm, % Vertical margin
	letterpaper, % Paper size, change to letterpaper for US letter size
	%showframe, % Uncomment to show how the type block is set on the page -- typically for debugging
}

%----------------------------------------------------------------------------------------
%	COLUMN LAYOUT
%----------------------------------------------------------------------------------------

\usepackage{paracol} % Required for creating multi-column layouts that can span pages automatically

\columnratio{0.55,0.45} % The relative ratios of the two columns in the CV

\setlength\columnsep{0.05\textwidth} % Specify the amount of space between the columns

%----------------------------------------------------------------------------------------
%	FONTS
%----------------------------------------------------------------------------------------

\usepackage{fontspec} % Required for specifying custom fonts under XeLaTeX

\setmainfont{EBGaramond}[ % Make the default font EBGaramond
Path=fonts/, % The font is provided with the template in the fonts folder
UprightFont=*-Regular.ttf,
BoldFont=*-Bold.ttf,
BoldItalicFont=*-BoldItalic.ttf,
ItalicFont=*-Italic.ttf,
SmallCapsFont=*-SC.ttf
]

\newfontfamily\cvtextfont[Path=fonts/]{freebooterscript} % Create a new font family for the cursive font Freebooter Script, provided with the template in the fonts folder

\newfontfamily\cvlangfont{Courier New} % Create a new font family for programming languages

\newfontfamily{\FA}[Path=fonts/]{FontAwesome} % Create a new font family for FontAwesome icons, provided with the template in the fonts folder
\input{fonts/fontawesomesymbols-xeluatex.tex} % Load a file to create shortcuts to the icons, see icon examples and their shortcuts in fontawesome.pdf in the fonts folder

\usepackage[sf,scale=0.95]{libertine} % Load Libertine as a \sffamily font for sans serif titles

%----------------------------------------------------------------------------------------
%	COLOURS AND LINKS
%----------------------------------------------------------------------------------------

\usepackage[usenames,svgnames]{xcolor} % Allows the definition and use of custom colours

\definecolor{text}{HTML}{2b2b2b} % Main document font colour, off-black
\definecolor{headings}{HTML}{701112} % Dark red colour for headings
\definecolor{shade}{HTML}{F5DD9D} % Peach colour for the contact information box
\definecolor{linkcolor}{HTML}{0000ff} % 25% desaturated headings colour for links
\definecolor{python}{HTML}{0E5484}
\definecolor{maingray}{HTML}{B9B9B9}
\definecolor{proglang}{HTML}{f2f2f2}
\definecolor{matlab}{HTML}{e16737}
\definecolor{cuda}{HTML}{3A4E3A}
\definecolor{rpackage}{HTML}{198CE7}
\definecolor{qt}{HTML}{3D6117}
% Other colour options: shade=B9D7D9 and linkcolor=A40000; shade=D4D7FE and linkcolor=FF0080

% For preset colours that can be used by their names without having to define them, see: https://www.latextemplates.com/svgnames-colors

\color{text} % Set the default text colour for the whole document to the colour defined as 'text' above

%------------------------------------------------

\usepackage{hyperref} % Required for links

\hypersetup{colorlinks, breaklinks, urlcolor=linkcolor, linkcolor=linkcolor} % Set up links and their colours


%----------------------------------------------------------------------------------------
%	HEADERS & FOOTERS
%----------------------------------------------------------------------------------------

\usepackage{fancyhdr} % Required for customising headers and footers

\pagestyle{fancy} % Enable custom headers and footers

\fancyhf{} % This suppresses all headers and footers by default, add headers and footers in the template file as per the example

\renewcommand{\headrulewidth}{0pt} % Remove the default rule under the header

%----------------------------------------------------------------------------------------
%	SECTIONS
%----------------------------------------------------------------------------------------

\usepackage[nobottomtitles*]{titlesec} % Required for modifying sections, the nobottomtitles* is required for pushing section titles to the next page when they are close to the bottom of the page

\renewcommand{\bottomtitlespace}{0.1\textheight} % Modify the minimal space required from the bottom margin not to move the title to the next page

\titleformat{\section}{\color{headings}\scshape\LARGE\raggedright}{}{0em}{}[\color{black}\titlerule] % Define the \section format

\titlespacing{\section}{0pt}{0pt}{8pt} % Spacing around section titles, the order is: left, before and after

%----------------------------------------------------------------------------------------
%	GRAPHICS DEFINITIONS
%---------------------------------------------------------------------------------------- 

\usepackage{tikz} % Required for creating the plots
\usetikzlibrary{shapes, backgrounds}
\tikzset{x=1cm, y=1cm} % Default tikz units

% Command to vertically centre adjacent content
\newcommand{\vcenteredhbox}[1]{% The only parameter is for the content to centre
	\begingroup%
		\setbox0=\hbox{#1}\parbox{\wd0}{\box0}%
	\endgroup%
}

%----------------------------------------------------------------------------------------
%	CHARTS
%---------------------------------------------------------------------------------------- 

\newcounter{barcount}

% Environment to hold a new bar chart
\newenvironment{barchart}[1]{ % The only parameter is the maximum bar width, in cm
	\newcommand{\barwidth}{0.35}
	\newcommand{\barsep}{0.2}
	
	% Command to add a bar to the bar chart
	\newcommand{\baritem}[2]{ % The first argument is the bar label and the second is the percentage the current bar should take up of the total width
		\pgfmathparse{##2}
		\let\perc\pgfmathresult
		
		\pgfmathparse{#1}
		\let\barsize\pgfmathresult
		
		\pgfmathparse{\barsize*##2/100}
		\let\barone\pgfmathresult
		
		\pgfmathparse{(\barwidth*\thebarcount)+(\barsep*\thebarcount)}
		\let\barx\pgfmathresult
		
		\filldraw[fill=shade, draw=none] (0,-\barx) rectangle (\barone,-\barx-\barwidth);
		
		\node [label=180:\colorbox{white}{\textcolor{text}{##1}}] at (0,-\barx-0.175) {};
		\addtocounter{barcount}{1}
	}
	\begin{tikzpicture}
		\setcounter{barcount}{0}
}{
	\end{tikzpicture}
}

%----------------------------------------------------------------------------------------
%	CUSTOM COMMANDS
%----------------------------------------------------------------------------------------

% Command for entering a new work position
\newcommand{\workposition}[7]{
	\multicolumn{2}{c}{
		\expandafter\ifstrequal\expandafter{#3}{}{}{\textbf{#3}} % Employer
		\expandafter\ifstrequal\expandafter{#4}{}{}{\hfill {\raggedright\textsc{#4}}} % Location
	}
	\expandafter\ifstrequal\expandafter{#3#4}{}{\\[-10pt]}{\\}
	\textsc{
		\expandafter\ifstrequal\expandafter{#1}{}{}{#1}
		\expandafter\ifstrequal\expandafter{#2}{}{}{\hspace{6pt}\footnotesize{(#2)}}
	} % Duration and conditional full time/part time text
	\expandafter\ifstrequal\expandafter{#5}{}{}{& {\textit{{#5}}}\\[4pt]} % Job title
	\expandafter\ifstrequal\expandafter{#6}{}{}{& #6} % Description
	\expandafter\ifstrequal\expandafter{#7}{}{}{\\[6pt]}
}

% Command for entering a separate qualification
\newcommand{\educationentry}[5]{
	\textsc{#1} & \textbf{#2} % Duration and degree
	\expandafter\ifstrequal\expandafter{#5}{}{}{& {\raggedright\textit{#5}}\\} % Institution
	\expandafter\ifstrequal\expandafter{#4}{}{}{& #4} % Department
	\expandafter\ifstrequal\expandafter{#3}{}{}{, {\small\textsc{#3}}\\} % Honours, achievements or distinctions (e.g. first class honours)
}

% Command for entering a separate table row -- used as a generic visual element for any section that uses a two column table
\newcommand{\tableentry}[3]{
	\textsc{#1} % for bullets
	& #2 % main text
	\expandafter\ifstrequal\expandafter{#3}{}{\\}{\\[6pt]} % First the heading, then content, then a conditional insertion of whitespace if the third parameter has any content in it
}

% Command for entering a separate table row -- used as a generic visual element for any section that uses a two column table
\newcommand{\tableentryraw}[3]{
	#1 & #2\expandafter\ifstrequal\expandafter{#3}{}{\\}{\\[6pt]} % First the heading, then content, then a conditional insertion of whitespace if the third parameter has any content in it
}

% Command for entering a separate table row -- used as a generic visual element for any section that uses a two column table
\newcommand{\tableentrythree}[4]{
	\textsc{#1} & #2 & \textsc{#3}\expandafter\ifstrequal\expandafter{#4}{}{\\}{\\[6pt]} % First the heading, then content, then the location, then a conditional insertion of whitespace if the third parameter has any content in it
}

% Command for entering a separate table row -- used as a generic visual element for any section that uses a two column table
\newcommand{\tableentryrawthree}[4]{
	\textsc{#1} & #2 & #3\expandafter\ifstrequal\expandafter{#4}{}{\\}{\\[6pt]} % First the heading, then content, then the location, then a conditional insertion of whitespace if the third parameter has any content in it
}


% Command for entering a long-form description where there is a title on one line and a paragraph description below it
\newcommand{\longformdescription}[2]{
	\textit{#1}\\[3pt]
	#2\medskip
}

% Command for entering a long-form description where there is a title on one line and a paragraph description below it
\newcommand{\longformdescriptiontight}[2]{
	\textit{#1}
	#2\medskip
}

% Command for entering a publication in long-form format
\newcommand{\longformpublication}[1]{
	#1\medskip
}

% Command for entering a publication as a short DOI (digital object identifier) string to the publication, a link is automatically created
\newcommand{\doipublication}[4]{
	#1 & % Year
	\href{http://dx.doi.org/#2}{\expandafter\ifstrequal\expandafter{#3}{firstauthor}{\textbf{doi:#2}}{doi:#2}}% DOI string and if "firstauthor" is entered for the 3rd argument, this makes the DOI string bold indicating a first author publication
	\expandafter\ifstrequal\expandafter{#4}{}{\\}{\\[3pt]} % Conditional insertion of whitespace if the 4th parameter has any content in it
}

% Command for creating skill level plots
\newcommand{\skilllevel}[2]{%
    \textcolor{black}{\textbf{#1}}~~~~~~\hfill
    \foreach \x in {1,...,10}{%
      \space{\ifnumgreater{\x}{#2}{\color{maingray}\faCircleO}{\color{headings}\faDotCircleO}}}\par%
}

% Define custom commands for CV info
\newcommand{\cvdate}[1]{\renewcommand{\cvdate}{#1}}
\newcommand{\cvmail}[1]{\renewcommand{\cvmail}{#1}}
\newcommand{\cvnumberphone}[1]{\renewcommand{\cvnumberphone}{#1}}
\newcommand{\cvaddress}[1]{\renewcommand{\cvaddress}{#1}}
\newcommand{\cvsite}[1]{\renewcommand{\cvsite}{#1}}
\newcommand{\aboutme}[1]{\renewcommand{\aboutme}{#1}}
\newcommand{\education}[1]{\renewcommand{\education}{#1}}
\newcommand{\references}[1]{\renewcommand{\references}{#1}}
\newcommand{\awards}[1]{\renewcommand{\awards}{#1}}
\newcommand{\profilepic}[1]{\renewcommand{\profilepic}{#1}}
\newcommand{\cvname}[1]{\renewcommand{\cvname}{#1}}
\newcommand{\cvjobtitle}[1]{\renewcommand{\cvjobtitle}{#1}}
\newcommand{\cvgithub}[1]{\renewcommand{\cvgithub}{#1}}
\newcommand{\cvlinkedin}[1]{\renewcommand{\cvlinkedin}{#1}}
\newcommand{\cvdescribe}[1]{\renewcommand{\cvdescribe}{#1}}
\newcommand{\cvskills}[1]{\renewcommand{\cvskills}{#1}}
\newcommand{\cvhobbies}[1]{\renewcommand{\cvhobbies}{#1}}
\newcommand{\cvreferences}[1]{\renewcommand{\cvreferences}{#1}}
\newcommand{\cvcourses}[1]{\renewcommand{\cvcourses}{#1}}

% Command to create the rounded boxes around the first three letters of section titles
\newcommand*\round[2]{%
	\begin{tikzpicture}[baseline=(char.base)]
		\node[align=center,anchor=north west, draw,rectangle, rounded corners=2.5mm, inner sep=1.6pt, minimum size=5.5mm, text centered, minimum height = 5mm, text height=2mm, fill=#2,#2,text=black](char){#1};%
	\end{tikzpicture}
}

% Command for creating a big colored dot
\newcommand{\programlang}[2]{
	\round{
		{#1$\bullet$}~~{\footnotesize\cvlangfont#2}
	}{proglang}
} % Include the file that specifies the document structure

% Headers and footers can be added with the \lhead{} \rhead{} \lfoot{} \rfoot{} commands
% Example right footer:
%\rfoot{\color{headings}{\sffamily Last update: \today. Typeset with Xe\LaTeX}}

%----------------------------------------------------------------------------------------
%	 PERSONAL INFORMATION
%----------------------------------------------------------------------------------------

% If you don't need one or more of the below, just remove the content leaving the command, e.g. \cvnumberphone{}

\profilepic{} % Profile picture

\cvname{Khalil Al Handawi, PhD} % Your name
\cvjobtitle{Engineer, designer, and researcher} % Job title/career

\cvdate{} % Date of birth
\cvaddress{Montr\'{e}al Qu\'{e}bec, Canada} % Short address/location, use \newline if more than 1 line is required
\cvnumberphone{+1 (514) 572-7367} % Phone number
\cvsite{khbalhandawi.github.io} % Personal website
\cvmail{khalil.alhandawi@mail.mcgill.ca} % Email address
\cvgithub{github.com/khbalhandawi} % GitHub
\cvlinkedin{linkedin.com/in/khbalhandawi} % LinkedIn

\aboutme{I believe that physics and artificial intelligence should be two sides of the same coin. One cannot exist without the other. How? By cross-validation. In this way, the toughest physics and mathematics problems can be solved! This philosophy is what drives my research.} % To have no About Me section, just remove all the text and leave \aboutme{}

\begin{document}

\begin{paracol}{2} % Begin the multi-column environment

%----------------------------------------------------------------------------------------
%	NAME AND CURRICULUM VITAE TEXT
%----------------------------------------------------------------------------------------

\parbox[top][0.1\textheight][c]{\linewidth}{ % Parbox to hold the author name and CV text; fixed height to match the coloured box to the right, centred vertically and full line width
	\vspace{0.0\textheight} % Reduce whitespace above the parbox to separate it from the main content
	%----------------------------------------------------------------------------------------
	%	ADDRESS
	%----------------------------------------------------------------------------------------
	\today\\[6pt]
	Meta\\
	Fremont, Montr\'{e}al, Canada\\
	Re: Postdoctoral Researcher, Artificial Intelligence | Chercheur ou chercheuse postdoctoral(e)\\[6pt] \medskip
}

%----------------------------------------------------------------------------------------

\switchcolumn % Switch to the next paracol column

%----------------------------------------------------------------------------------------
%	COLOURED CONTACT DETAILS BOX
%----------------------------------------------------------------------------------------

\parbox[top][0.1\textheight][c]{\linewidth}{ % Parbox to hold the colour box; fixed height to match the name/CV text to the left, centred vertically and full line width
	\vspace{0\textheight} % Reduce whitespace above the parbox to separate it from the main content
	\colorbox{shade}{ % Create the coloured box
		\begin{supertabular}{p{0.05\linewidth}|p{0.775\linewidth}} % Start a table with two columns, the table will ensure everything is aligned
			\ifthenelse{\equal{\cvdate}{}}{}{\raisebox{-1pt}{\faInfo} & \cvdate \\}
			\ifthenelse{\equal{\cvaddress}{}}{}{\raisebox{-1pt}{\faHome} & \cvaddress \\} % Address
			\ifthenelse{\equal{\cvaddress}{}}{}{\raisebox{-1pt}{\faPhone} & \cvnumberphone \\} % Phone number
			\ifthenelse{\equal{\cvmail}{}}{}{\raisebox{0pt}{\small\faEnvelope} & \href{mailto:\cvmail}{\cvmail} \\} % Email address
			\ifthenelse{\equal{\cvsite}{}}{}{\raisebox{-1pt}{\faGlobe} & \href{https://\cvsite}{\cvsite} \\} % GitHub profile
			\ifthenelse{\equal{\cvlinkedin}{}}{}{\raisebox{-1pt}{\faLinkedinSquare} & \href{https://\cvlinkedin}{\cvlinkedin} \\} % LinkedIn profile
			% See fontawesome.pdf in the fonts folder for all icons you can use
		\end{supertabular}
	}
}

\end{paracol}

\medskip % Extra whitespace before the next section
\rule[0pt]{\textwidth}{1pt}\\

%----------------------------------------------------------------------------------------
%	MAIN BODY
%----------------------------------------------------------------------------------------
Dear Talent acquisition manager,

\medskip % Extra whitespace before the next section 
I have always been fascinated by multidisciplinary engineering and design and have been conducting research in this field for about a decade. My research suggests that multidisciplinary design optimization and transparent communication and coordination between various subsystems is key to an efficient product platform able to not only perform its intended function but also adapt to new and unforeseen requirements. The Meta family has a wide range of product platforms involving hundreds of specialized teams and it is of great interest to me to help coordinate their efforts and provide the world with unprecedented connectivity and technology solutions.

\medskip % Extra whitespace before the next section

My lab, where I did my doctoral studies specializes in design optimization and machine learning for coordinating the engineering activities of multidisciplinary teams and organizations. I believe that my research experience is well-aligned with the work being done at Meta R\&D and the role that I would be filling as a postdoctoral researcher in artificial intelligence. My PhD dissertation focused on design automation of aerospace structures and the use of optimization and advanced data visualizations (parallel coordinates and hypersurfaces) to explore thousands of design alternatives. I also used Monte-Carlo simulation techniques to estimate various design attributes such as reliability against uncertain loads and requirements. The design problems that I solved involved multiple interacting disciplinary analyses (structural, additive manufacturing, and lifecycle cost analyses). I also used surrogate models and supervised learning techniques to create models in lieu of complex simulation models to reduce the computational effort needed to conduct certain analyses.

\medskip % Extra whitespace before the next section

This work was translated into industrial practice during my research visits to GKN aeroengine systems where I hosted engineering workshops and presentations related to my research into design optimization and uncertainty modeling. My collective PhD experience will allow me to develop optimization frameworks for various various product and prototyping platforms and conduct the necessary sensitivity analyses (using post-optimality analysis or multi-objective optimization) to get the most out of Meta's technology and products.

\medskip % Extra whitespace before the next section

Recently, after obtaining my PhD degree, I have been working with recurrent neural networks (RNNs) and encoder/decoder paradigms (Seq2Seq) models for now-casting COVID-19 pandemic trajectories for use in hospitals across Qu{\'e}bec and have come up with an innovative way to tune such models (hyperparameter optimization) using stochastic optimization methods. I believe that deep learning paradigms based on RNNs and DNNs can compliment multidisciplinary analyses by providing computationally low-cost predictions for certain subsystem analyses and can help coordinate the efforts of multidisciplinary teams by communicating data-driven models and response surfaces of the entire design space instead of discrete design solutions. I also lectured the engineering systems optimization course at McGill University and conceived a multidisciplinary optimization (MDO) project for the students with applications to supersonic jet design to teach the students about the importance of non-hierarchal MDO formulations for tightly coupled design problems.

\medskip % Extra whitespace before the next section

I also have massive amounts of experience with scalable scientific computing and is able to leverage and use parallel computing libraries and APIs such as openMP and CUDA across multiple programming languages such as MATLAB, Python, and C/C++ which will be invaluable when it comes to simulation model deployment and integration into Meta's software systems. I am also adept at maintaining and deploying code repositories using source control which will come in handy when partnering with different teams and subject matter experts on prototyping logistics and various other product platforms.

\medskip % Extra whitespace before the next section

I also conducted research visits to my alma mater, Khalifa University, where I worked on sensor fusion projects related to Unmanned aerial vehicles (UAVs). These drones are difficult to control using positioning systems (such as GPS and motion capture systems) due to their high latencies and low frequency signals. I managed to develop a sensor fusion algorithm for combining high frequency inertial measurement unit (IMU) signals with low frequency motion capture signals for attitude estimation and subsequent use for control applications. I also repurposed a commercial UAV (DJI Wind4) for firefighting applications by reverse engineering it and adding an extinguishant payload and delivery subsystem. This experience provided me with experience in working with electrical and control engineers to deliver a product that meets the needs of the client (autonomous firefighting technology) and should reflect on my ability to work with various teams and disciplines within the Meta family.

\medskip % Extra whitespace before the next section

Although I am happy with my current job working as a researcher at McGill and coming up with my own research ideas, I wanted to explore something different that I could only find at Meta. That is the opportunity to work with a multidisciplinary team and bring people's ideas together. I believe that my strong mathematical and simulation skills coupled with my experience in aerospace engineering, automotive engineering, structural simulation, engineering design, and software development will add a lot of value to optimization R\&D at Meta and help us both realize our vision of a better tomorrow for everyone around the world. I hope you enjoy going through my profile and my projects on my website (\href{https://khbalhandawi.github.io/projects/}{https://khbalhandawi.github.io/projects/}) and I hope there is mutual interest in an opportunity to work together.

\medskip % Extra whitespace before the next section

Yours sincerely,

\medskip % Extra whitespace before the next section

Khalil Al Handawi

\begin{figure*}[h]
	\includegraphics[width=0.15\textwidth]{Signiture.png}
\end{figure*}

\medskip % Extra whitespace before the next section

%----------------------------------------------------------------------------------------

\end{document}