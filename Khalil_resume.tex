%%%%%%%%%%%%%%%%%%%%%%%%%%%%%%%%%%%%%%%%%
% Freeman Curriculum Vitae
% XeLaTeX Template
% Version 2.0 (19/3/2018)
%
% This template originates from:
% http://www.LaTeXTemplates.com
%
% Authors:
% Vel (vel@LaTeXTemplates.com)
% Alessandro Plasmati
%
% License:
% CC BY-NC-SA 3.0 (http://creativecommons.org/licenses/by-nc-sa/3.0/)
%
%!TEX program = xelatex
% NOTICE: This template must be compiled with XeLaTeX, the line above should
% ensure this happens automatically but if it doesn't you will need to specify
% XeLaTeX as the engine in your editor or script
%
%%%%%%%%%%%%%%%%%%%%%%%%%%%%%%%%%%%%%%%%%

%----------------------------------------------------------------------------------------
%	PACKAGES AND OTHER DOCUMENT CONFIGURATIONS
%----------------------------------------------------------------------------------------

\documentclass[10pt]{article} % Font size, can be: 10pt, 11pt or 12pt

%%%%%%%%%%%%%%%%%%%%%%%%%%%%%%%%%%%%%%%%%
% Freeman Curriculum Vitae
% Structure Specification File
% Version 1.0 (19/3/2018)
%
% This template originates from:
% http://www.LaTeXTemplates.com
%
% Authors:
% Vel (vel@LaTeXTemplates.com)
% Alessandro Plasmati
%
% License:
% CC BY-NC-SA 3.0 (http://creativecommons.org/licenses/by-nc-sa/3.0/)
% 
%%%%%%%%%%%%%%%%%%%%%%%%%%%%%%%%%%%%%%%%%

%----------------------------------------------------------------------------------------
%	PACKAGES AND OTHER DOCUMENT CONFIGURATIONS
%----------------------------------------------------------------------------------------

\usepackage{etoolbox} % Required for conditional statements

\setlength\parindent{0pt} % Stop paragraph indentation

\usepackage{supertabular} % Required for the supertabular environment which allows tables to span multiple pages so sections with tables correctly wrap across pages

\usepackage{array} % for custom alignment if supertabular columns

\usepackage{ifthen} % For creating conditional entries

\usepackage{enumitem} % for formatting bulleted lists and reducing whitespace before bullet list

%----------------------------------------------------------------------------------------
%	DOCUMENT MARGINS
%----------------------------------------------------------------------------------------

\usepackage{geometry} % Required for adjusting page dimensions and margins

\geometry{
	hmargin=1.5cm, % Horizontal margin
	vmargin=1.75cm, % Vertical margin
	letterpaper, % Paper size, change to letterpaper for US letter size
	%showframe, % Uncomment to show how the type block is set on the page -- typically for debugging
}

%----------------------------------------------------------------------------------------
%	COLUMN LAYOUT
%----------------------------------------------------------------------------------------

\usepackage{paracol} % Required for creating multi-column layouts that can span pages automatically

\columnratio{0.55,0.45} % The relative ratios of the two columns in the CV

\setlength\columnsep{0.05\textwidth} % Specify the amount of space between the columns

%----------------------------------------------------------------------------------------
%	FONTS
%----------------------------------------------------------------------------------------

\usepackage{fontspec} % Required for specifying custom fonts under XeLaTeX

\setmainfont{EBGaramond}[ % Make the default font EBGaramond
Path=fonts/, % The font is provided with the template in the fonts folder
UprightFont=*-Regular.ttf,
BoldFont=*-Bold.ttf,
BoldItalicFont=*-BoldItalic.ttf,
ItalicFont=*-Italic.ttf,
SmallCapsFont=*-SC.ttf
]

\newfontfamily\cvtextfont[Path=fonts/]{freebooterscript} % Create a new font family for the cursive font Freebooter Script, provided with the template in the fonts folder

\newfontfamily\cvlangfont{Courier New} % Create a new font family for programming languages

\newfontfamily{\FA}[Path=fonts/]{FontAwesome} % Create a new font family for FontAwesome icons, provided with the template in the fonts folder
\input{fonts/fontawesomesymbols-xeluatex.tex} % Load a file to create shortcuts to the icons, see icon examples and their shortcuts in fontawesome.pdf in the fonts folder

\usepackage[sf,scale=0.95]{libertine} % Load Libertine as a \sffamily font for sans serif titles

%----------------------------------------------------------------------------------------
%	COLOURS AND LINKS
%----------------------------------------------------------------------------------------

\usepackage[usenames,svgnames]{xcolor} % Allows the definition and use of custom colours

\definecolor{text}{HTML}{2b2b2b} % Main document font colour, off-black
\definecolor{headings}{HTML}{701112} % Dark red colour for headings
\definecolor{shade}{HTML}{F5DD9D} % Peach colour for the contact information box
\definecolor{linkcolor}{HTML}{0000ff} % 25% desaturated headings colour for links
\definecolor{python}{HTML}{0E5484}
\definecolor{maingray}{HTML}{B9B9B9}
\definecolor{proglang}{HTML}{f2f2f2}
\definecolor{matlab}{HTML}{e16737}
\definecolor{cuda}{HTML}{3A4E3A}
\definecolor{rpackage}{HTML}{198CE7}
\definecolor{qt}{HTML}{3D6117}
% Other colour options: shade=B9D7D9 and linkcolor=A40000; shade=D4D7FE and linkcolor=FF0080

% For preset colours that can be used by their names without having to define them, see: https://www.latextemplates.com/svgnames-colors

\color{text} % Set the default text colour for the whole document to the colour defined as 'text' above

%------------------------------------------------

\usepackage{hyperref} % Required for links

\hypersetup{colorlinks, breaklinks, urlcolor=linkcolor, linkcolor=linkcolor} % Set up links and their colours


%----------------------------------------------------------------------------------------
%	HEADERS & FOOTERS
%----------------------------------------------------------------------------------------

\usepackage{fancyhdr} % Required for customising headers and footers

\pagestyle{fancy} % Enable custom headers and footers

\fancyhf{} % This suppresses all headers and footers by default, add headers and footers in the template file as per the example

\renewcommand{\headrulewidth}{0pt} % Remove the default rule under the header

%----------------------------------------------------------------------------------------
%	SECTIONS
%----------------------------------------------------------------------------------------

\usepackage[nobottomtitles*]{titlesec} % Required for modifying sections, the nobottomtitles* is required for pushing section titles to the next page when they are close to the bottom of the page

\renewcommand{\bottomtitlespace}{0.1\textheight} % Modify the minimal space required from the bottom margin not to move the title to the next page

\titleformat{\section}{\color{headings}\scshape\LARGE\raggedright}{}{0em}{}[\color{black}\titlerule] % Define the \section format

\titlespacing{\section}{0pt}{0pt}{8pt} % Spacing around section titles, the order is: left, before and after

%----------------------------------------------------------------------------------------
%	GRAPHICS DEFINITIONS
%---------------------------------------------------------------------------------------- 

\usepackage{tikz} % Required for creating the plots
\usetikzlibrary{shapes, backgrounds}
\tikzset{x=1cm, y=1cm} % Default tikz units

% Command to vertically centre adjacent content
\newcommand{\vcenteredhbox}[1]{% The only parameter is for the content to centre
	\begingroup%
		\setbox0=\hbox{#1}\parbox{\wd0}{\box0}%
	\endgroup%
}

%----------------------------------------------------------------------------------------
%	CHARTS
%---------------------------------------------------------------------------------------- 

\newcounter{barcount}

% Environment to hold a new bar chart
\newenvironment{barchart}[1]{ % The only parameter is the maximum bar width, in cm
	\newcommand{\barwidth}{0.35}
	\newcommand{\barsep}{0.2}
	
	% Command to add a bar to the bar chart
	\newcommand{\baritem}[2]{ % The first argument is the bar label and the second is the percentage the current bar should take up of the total width
		\pgfmathparse{##2}
		\let\perc\pgfmathresult
		
		\pgfmathparse{#1}
		\let\barsize\pgfmathresult
		
		\pgfmathparse{\barsize*##2/100}
		\let\barone\pgfmathresult
		
		\pgfmathparse{(\barwidth*\thebarcount)+(\barsep*\thebarcount)}
		\let\barx\pgfmathresult
		
		\filldraw[fill=shade, draw=none] (0,-\barx) rectangle (\barone,-\barx-\barwidth);
		
		\node [label=180:\colorbox{white}{\textcolor{text}{##1}}] at (0,-\barx-0.175) {};
		\addtocounter{barcount}{1}
	}
	\begin{tikzpicture}
		\setcounter{barcount}{0}
}{
	\end{tikzpicture}
}

%----------------------------------------------------------------------------------------
%	CUSTOM COMMANDS
%----------------------------------------------------------------------------------------

% Command for entering a new work position
\newcommand{\workposition}[7]{
	\multicolumn{2}{c}{
		\expandafter\ifstrequal\expandafter{#3}{}{}{\textbf{#3}} % Employer
		\expandafter\ifstrequal\expandafter{#4}{}{}{\hfill {\raggedright\textsc{#4}}} % Location
	}
	\expandafter\ifstrequal\expandafter{#3#4}{}{\\[-10pt]}{\\}
	\textsc{
		\expandafter\ifstrequal\expandafter{#1}{}{}{#1}
		\expandafter\ifstrequal\expandafter{#2}{}{}{\hspace{6pt}\footnotesize{(#2)}}
	} % Duration and conditional full time/part time text
	\expandafter\ifstrequal\expandafter{#5}{}{}{& {\textit{{#5}}}\\[4pt]} % Job title
	\expandafter\ifstrequal\expandafter{#6}{}{}{& #6} % Description
	\expandafter\ifstrequal\expandafter{#7}{}{}{\\[6pt]}
}

% Command for entering a separate qualification
\newcommand{\educationentry}[5]{
	\textsc{#1} & \textbf{#2} % Duration and degree
	\expandafter\ifstrequal\expandafter{#5}{}{}{& {\raggedright\textit{#5}}\\} % Institution
	\expandafter\ifstrequal\expandafter{#4}{}{}{& #4} % Department
	\expandafter\ifstrequal\expandafter{#3}{}{}{, {\small\textsc{#3}}\\} % Honours, achievements or distinctions (e.g. first class honours)
}

% Command for entering a separate table row -- used as a generic visual element for any section that uses a two column table
\newcommand{\tableentry}[3]{
	\textsc{#1} % for bullets
	& #2 % main text
	\expandafter\ifstrequal\expandafter{#3}{}{\\}{\\[6pt]} % First the heading, then content, then a conditional insertion of whitespace if the third parameter has any content in it
}

% Command for entering a separate table row -- used as a generic visual element for any section that uses a two column table
\newcommand{\tableentryraw}[3]{
	#1 & #2\expandafter\ifstrequal\expandafter{#3}{}{\\}{\\[6pt]} % First the heading, then content, then a conditional insertion of whitespace if the third parameter has any content in it
}

% Command for entering a separate table row -- used as a generic visual element for any section that uses a two column table
\newcommand{\tableentrythree}[4]{
	\textsc{#1} & #2 & \textsc{#3}\expandafter\ifstrequal\expandafter{#4}{}{\\}{\\[6pt]} % First the heading, then content, then the location, then a conditional insertion of whitespace if the third parameter has any content in it
}

% Command for entering a separate table row -- used as a generic visual element for any section that uses a two column table
\newcommand{\tableentryrawthree}[4]{
	\textsc{#1} & #2 & #3\expandafter\ifstrequal\expandafter{#4}{}{\\}{\\[6pt]} % First the heading, then content, then the location, then a conditional insertion of whitespace if the third parameter has any content in it
}


% Command for entering a long-form description where there is a title on one line and a paragraph description below it
\newcommand{\longformdescription}[2]{
	\textit{#1}\\[3pt]
	#2\medskip
}

% Command for entering a long-form description where there is a title on one line and a paragraph description below it
\newcommand{\longformdescriptiontight}[2]{
	\textit{#1}
	#2\medskip
}

% Command for entering a publication in long-form format
\newcommand{\longformpublication}[1]{
	#1\medskip
}

% Command for entering a publication as a short DOI (digital object identifier) string to the publication, a link is automatically created
\newcommand{\doipublication}[4]{
	#1 & % Year
	\href{http://dx.doi.org/#2}{\expandafter\ifstrequal\expandafter{#3}{firstauthor}{\textbf{doi:#2}}{doi:#2}}% DOI string and if "firstauthor" is entered for the 3rd argument, this makes the DOI string bold indicating a first author publication
	\expandafter\ifstrequal\expandafter{#4}{}{\\}{\\[3pt]} % Conditional insertion of whitespace if the 4th parameter has any content in it
}

% Command for creating skill level plots
\newcommand{\skilllevel}[2]{%
    \textcolor{black}{\textbf{#1}}~~~~~~\hfill
    \foreach \x in {1,...,10}{%
      \space{\ifnumgreater{\x}{#2}{\color{maingray}\faCircleO}{\color{headings}\faDotCircleO}}}\par%
}

% Define custom commands for CV info
\newcommand{\cvdate}[1]{\renewcommand{\cvdate}{#1}}
\newcommand{\cvmail}[1]{\renewcommand{\cvmail}{#1}}
\newcommand{\cvnumberphone}[1]{\renewcommand{\cvnumberphone}{#1}}
\newcommand{\cvaddress}[1]{\renewcommand{\cvaddress}{#1}}
\newcommand{\cvsite}[1]{\renewcommand{\cvsite}{#1}}
\newcommand{\aboutme}[1]{\renewcommand{\aboutme}{#1}}
\newcommand{\education}[1]{\renewcommand{\education}{#1}}
\newcommand{\references}[1]{\renewcommand{\references}{#1}}
\newcommand{\awards}[1]{\renewcommand{\awards}{#1}}
\newcommand{\profilepic}[1]{\renewcommand{\profilepic}{#1}}
\newcommand{\cvname}[1]{\renewcommand{\cvname}{#1}}
\newcommand{\cvjobtitle}[1]{\renewcommand{\cvjobtitle}{#1}}
\newcommand{\cvgithub}[1]{\renewcommand{\cvgithub}{#1}}
\newcommand{\cvlinkedin}[1]{\renewcommand{\cvlinkedin}{#1}}
\newcommand{\cvdescribe}[1]{\renewcommand{\cvdescribe}{#1}}
\newcommand{\cvskills}[1]{\renewcommand{\cvskills}{#1}}
\newcommand{\cvhobbies}[1]{\renewcommand{\cvhobbies}{#1}}
\newcommand{\cvreferences}[1]{\renewcommand{\cvreferences}{#1}}
\newcommand{\cvcourses}[1]{\renewcommand{\cvcourses}{#1}}

% Command to create the rounded boxes around the first three letters of section titles
\newcommand*\round[2]{%
	\begin{tikzpicture}[baseline=(char.base)]
		\node[align=center,anchor=north west, draw,rectangle, rounded corners=2.5mm, inner sep=1.6pt, minimum size=5.5mm, text centered, minimum height = 5mm, text height=2mm, fill=#2,#2,text=black](char){#1};%
	\end{tikzpicture}
}

% Command for creating a big colored dot
\newcommand{\programlang}[2]{
	\round{
		{#1$\bullet$}~~{\footnotesize\cvlangfont#2}
	}{proglang}
} % Include the file that specifies the document structure

% Headers and footers can be added with the \lhead{} \rhead{} \lfoot{} \rfoot{} commands
% Example right footer:
%\rfoot{\color{headings}{\sffamily Last update: \today. Typeset with Xe\LaTeX}}

%----------------------------------------------------------------------------------------
%	 PERSONAL INFORMATION
%----------------------------------------------------------------------------------------

% If you don't need one or more of the below, just remove the content leaving the command, e.g. \cvnumberphone{}

\profilepic{} % Profile picture

\cvname{Khalil Al Handawi, PhD} % Your name
\cvjobtitle{Engineer, designer, and researcher} % Job title/career

\cvdate{} % Date of birth
\cvaddress{Montr\'{e}al Qu\'{e}bec, Canada} % Short address/location, use \newline if more than 1 line is required
\cvnumberphone{+1 (514) 572-7367} % Phone number
\cvsite{sol.research.mcgill.ca} % Personal website
\cvmail{khalil.alhandawi@mail.mcgill.ca} % Email address
\cvgithub{github.com/khbalhandawi} % GitHub
\cvlinkedin{linkedin.com/in/khbalhandawi} % LinkedIn


%----------------------------------------------------------------------------------------

\begin{document}

%----------------------------------------------------------------------------------------
%	NAME AND CURRICULUM VITAE TEXT
%----------------------------------------------------------------------------------------

\parbox[top][0.06\textheight][c]{\linewidth}{ % Parbox to hold the author name and CV text; fixed height to match the coloured box to the right, centred vertically and full line width
	\vspace{0.005\textheight} % Reduce whitespace above the parbox to separate it from the main content
	\centering % Centre text
	{\sffamily\LARGE \cvname}\\\medskip % Your name
	{\sffamily\Large\color{headings}\textit \cvjobtitle}\\ % Your profession
	% {\Huge\color{headings}\cvtextfont Curriculum Vitae}
}

%----------------------------------------------------------------------------------------
%	 SIDEBAR DEFINITIONS
%----------------------------------------------------------------------------------------

\setlength{\TPHorizModule}{1cm} % Left margin
\setlength{\TPVertModule}{1cm} % Top margin

%----------------------------------------------------------------------------------------
%	 ABOUT ME
%----------------------------------------------------------------------------------------

\aboutme{I believe that physics and artificial intelligence should be two sides of the same coin. One cannot exist without the other. How? By cross-validation. In this way, the toughest physics and mathematics problems can be solved! This philosophy is what drives my research.} % To have no About Me section, just remove all the text and leave \aboutme{}

%----------------------------------------------------------------------------------------
%	 EDUCATION
%----------------------------------------------------------------------------------------

\education{
	\begin{supertabular}{rl} % Start a table with two columns, the table will ensure everything is aligned

		%------------------------------------------------

		\educationentry{2017 -- 2020} % Duration
		{Doctor of Philosophy} % Degree
		{} % Honours, achievements or distinctions (e.g. first class honours)
		{Mechanical Engineering} % Department
		{McGill University} % Institution

		%------------------------------------------------

		\educationentry{2013 -- 2015} % Duration
		{Master of Science} % Degree
		{} % Honours, achievements or distinctions (e.g. first class honours)
		{Mechanical Engineering} % Department
		{Khalifa University} % Institution

		%------------------------------------------------

		\educationentry{2009 -- 2013} % Duration
		{Bachelor of Science} % Degree
		{First Class Honours} % Honours, achievements or distinctions (e.g. first class honours)
		{Mechanical Engineering} % Department
		{Khalifa University} % Institution

		%------------------------------------------------

	\end{supertabular}
}

%----------------------------------------------------------------------------------------
%	AWARDS
%----------------------------------------------------------------------------------------

% Example \tableentry{} command to add another line:

%\tableentry{Heading}{Content}{spaceafter}

% All 3 parameters must be supplied but any can be empty if you don't need them
% A "spaceafter" value in the third parameter will add some vertical space -- this is to be used between headings

\awards{
	%------------------------------------------------

	\begin{supertabular}{rl} % Start a table with two columns, the table will ensure everything is aligned

		%------------------------------------------------

		\tableentry{2018}{\textbf{Doctoral research award}}{}
		\tableentry{}{\textit{Fonds de Recherche du Qu\'{e}bec}}{}
		\tableentry{}{\textsc{56,000 CAD}}{spaceafter}

		%------------------------------------------------

		\tableentry{2017}{\textbf{McGill engineering doctoral award}}{}
		\tableentry{}{\textit{McGill University}}{}
		\tableentry{}{\textsc{96,000 CAD}}{spaceafter}
		%------------------------------------------------

	\end{supertabular}
}

%----------------------------------------------------------------------------------------
%	 REFERENCES
%----------------------------------------------------------------------------------------

\references{
	\begin{supertabular}{rl} % Start a table with two columns, the table will ensure everything is aligned

		%------------------------------------------------

		\tableentry{}{\textbf{Prof. Michael Kokkolaras}}{spaceafter}
		\tableentry{Position}{Associate Professor}{}
		\tableentry{Employer}{\href{https://www.mcgill.ca/mecheng/}{Department of Mechanical}}{}
		\tableentry{}{\href{https://www.mcgill.ca/}{Engineering, \textit{McGill Univeristy}}}{spaceafter}
		\tableentry{Email}{\href{mailto:michael.kokkolaras@mcgill.ca}{michael.kokkolaras@mcgill.ca}}{}

		% %------------------------------------------------

		% \tableentry{}{}{} % Creates some additional whitespace between the references

		% %------------------------------------------------

		% \tableentry{}{\textbf{Prof. Nader Vahdati}}{spaceafter}
		% \tableentry{Position}{Associate Professor}{}
		% \tableentry{Employer}{\href{}{Department of Mechanical}}{}
		% \tableentry{}{\href{https://www.ku.ac.ae/}{Engineering, \textit{Khalifa Univeristy}}}{spaceafter}
		% \tableentry{Email}{\href{mailto:nader.vahdati@ku.ac.ae}{nader.vahdati@ku.ac.ae}}{}

		%------------------------------------------------

	\end{supertabular}
}

%----------------------------------------------------------------------------------------
%	 SKILLS
%----------------------------------------------------------------------------------------

% Skill bar section, each skill must have a value between 0 an 6 (float)
\skills{{Scientific computing/5.0},{Uncertainty quantification/4.5},{Software development/4.0},{{CAD/3D~modeling}/5.7},{Machine learning/4.8},{Optimization/5.0}}

%----------------------------------------------------------------------------------------

\makeprofile % Print the sidebar

%----------------------------------------------------------------------------------------
%	MAJOR RESEARCH PROJECT
%----------------------------------------------------------------------------------------

\vspace{-\baselineskip} % Standardise the whitespace after this section and before the next (the custom command adds too much otherwise)

\section{Research}

{\raggedright\textbf{``Optimization-driven set-based design for dynamic design requirements"}}
\textit{\center How do you design a component when the design requirements can change at any moment and without advance notice?}
That is the question my dissertations tries to answer. To do so, I came up with design metrics for qualitative descriptions such as flexibility and robustness. I used optimization, and machine learning to obtain thousands of designs. This is a \textbf{1000 fold} increase in the number of alternatives presented to clients. This culminated in a technology transfer at GKN aerospace.

\programlang{\color{python}}{python}~\programlang{\color{magenta}}{C++}~\programlang{\color{matlab}}{MATLAB}~\programlang{\color{rpackage}}{R}~\round{\href{https://github.com/khbalhandawi/DM_SBD_opt}{\small\faGlobe~Online open-source code}}{proglang}~\round{\href{https://www.chalmers.se/en/departments/ims/news/Pages/Optimizationdriven-setbased-design-.aspx}{\small\faGlobe~Online news article}}{proglang}

\medskip % Extra whitespace before the next section

{\raggedright\textbf{``Optimization of infectious disease prevention policies using agent-based modeling"}}
\textit{\center How can we apply the principles of design and decision-making to help bring the pandemic under control?}
To answer this question, I modeled how an infectious disease spreads in a small population. Diseases such as COVID-19 spread through social interaction. I programmed intelligent agents to model a complex social system. I used optimization to determine the critical amount of intervention necessary to keep the disease in check. The policies I obtained had a socio-economic impact that is \textbf{5 times less} than that of a complete lock-down.

\programlang{\color{magenta}}{C++}~\programlang{\color{cuda}}{CUDA}~\programlang{\color{python}}{python}~\programlang{\color{qt}}{Qt}~\round{\href{https://github.com/khbalhandawi/COVID_SIM}{\small\faGlobe~Online open-source code}}{proglang}

%----------------------------------------------------------------------------------------
%	WORK EXPERIENCE
%----------------------------------------------------------------------------------------

\section{Work Experience}

% Blank \workposition command to add another job:

%\workposition{} % Duration
%{} % FT/PT (full time or part time)
%{} % Employer
%{} % Job title
%{} % Description

% All 5 parameters must be supplied but any can be empty if you don't need them

%------------------------------------------------

\workposition{Current, from Jan 2021} % Duration
{} % FT/PT (full time or part time)
{Systems Optimization Lab, McGill University} % Employer
{Postdoctoral Researcher} % Job title
{
	$\bullet$ Built and implemented a COVID-19 predictive model in a time of uncertainty.\\
	$\bullet$ Came up with a project for students to understand multidisciplinary optimization.
} % Description

%------------------------------------------------

\workposition{Jan 2017 -- Dec 2020} % Duration
{FT} % FT/PT (full time or part time)
{McGill University} % Employer
{Research and teaching assistant} % Job title
{
	$\bullet$ Came up with new ways to teach programming skills to engineering students.\\
	$\bullet$ Used design optimization and set-based design to give designers a competitive edge.
}  % Description

%------------------------------------------------

\workposition{Summer 2017, 2018, 2019} % Duration
{PT} % FT/PT (full time or part time)
{GKN Aerospace Engine Systems} % Employer
{Visiting researcher} % Job title
{
	$\bullet$ Transfer academic research to the industry by providing training and workshops.\\
	$\bullet$ Collect information about industrial workflows to guide academic research.
}  % Description

%------------------------------------------------

\vspace{-\baselineskip} % Standardise the whitespace after this section and before the next (the custom command adds too much otherwise)

%----------------------------------------------------------------------------------------
%	COMPUTER SKILLS
%----------------------------------------------------------------------------------------

\section{Computer Skills}

\columnratio{0.6,0.4} % The relative ratios of the three columns in the references

\setlength\columnsep{0.01\textwidth} % Specify the amount of space between the columns

\begin{paracol}{2} % Begin the multi-column environment

	\begin{supertabular}{>{\raggedleft\arraybackslash}p{0.4\columnwidth}>{\raggedright\arraybackslash}p{0.6\columnwidth}} % Start a table with three columns, the table will ensure everything is aligned
		\tableentry{OS's ~~\faDesktop}{\faWindows ~~~\faLinux ~~~\faApple}{spaceafter}
		\tableentry{Libraries ~~\faCogs}{Qt, PyTorch, TensorFlow, CUDA, Intel MPI, OpenCL}{spaceafter}
		\tableentry{Source control ~~\faCodeFork}{Git, Perforce}{spaceafter}
		\tableentry{IDE's ~~\faListAlt}{VSCode, Xcode, Visual Studio}{}
	\end{supertabular}
		
	%----------------------------------------------------------------------------------------
	
	\switchcolumn % Switch to the next paracol column
	
	%----------------------------------------------------------------------------------------

	\begin{supertabular}{>{\raggedleft\arraybackslash}p{0.01\columnwidth}r} % Start a table with three columns, the table will ensure everything is aligned

		%------------------------------------------------
	
		\tableentry{}{\skilllevel{Python}{7}}{}
		\tableentry{}{\skilllevel{C++}{6}}{}
		\tableentry{}{\skilllevel{VB}{4}}{}
		\tableentry{}{\skilllevel{R}{3}}{}
		\tableentry{}{\skilllevel{Matlab}{7}}{}
		\tableentry{}{\skilllevel{Javascript}{4}}{}
	
		%------------------------------------------------
	
	\end{supertabular}

	%----------------------------------------------------------------------------------------
	
\end{paracol}

%----------------------------------------------------------------------------------------
%	SKILLS DESCRIPTION
%----------------------------------------------------------------------------------------

\section{Skills}

% Example \longformdescription{} command to add another section:

%\longformdescription{Heading}{Description}

%------------------------------------------------

\longformdescription{\faBullseye~~Goal Oriented}{I believe in action over long-winded discussions. I listen to everyone's viewpoints and use my judgement to immediately act based on consensus to achieve goals quickly and efficiently.}

\longformdescription{\faLineChart~~Passionate}{I have been interested in computers and video games for as long as I can remember. I love using my physics and engineering knowledge to bridge the gap between real and virtual computer worlds.}

\longformdescription{\faUniversalAccess~~Physical Dexterity}{I specialize in gymnastics and calethtinics training for sound body and mind necessary to maintain the focus needed for innovative problem-solving.}

%----------------------------------------------------------------------------------------

\end{document}
