%%%%%%%%%%%%%%%%%%%%%%%%%%%%%%%%%%%%%%%%%
% Freeman Curriculum Vitae
% XeLaTeX Template
% Version 2.0 (19/3/2018)
%
% This template originates from:
% http://www.LaTeXTemplates.com
%
% Authors:
% Vel (vel@LaTeXTemplates.com)
% Alessandro Plasmati
%
% License:
% CC BY-NC-SA 3.0 (http://creativecommons.org/licenses/by-nc-sa/3.0/)
%
%!TEX program = xelatex
% NOTICE: This template must be compiled with XeLaTeX, the line above should
% ensure this happens automatically but if it doesn't you will need to specify 
% XeLaTeX as the engine in your editor or script
% 
%%%%%%%%%%%%%%%%%%%%%%%%%%%%%%%%%%%%%%%%%

%----------------------------------------------------------------------------------------
%	PACKAGES AND OTHER DOCUMENT CONFIGURATIONS
%----------------------------------------------------------------------------------------

\documentclass[10pt]{article} % Font size, can be: 10pt, 11pt or 12pt

%%%%%%%%%%%%%%%%%%%%%%%%%%%%%%%%%%%%%%%%%
% Freeman Curriculum Vitae
% Structure Specification File
% Version 1.0 (19/3/2018)
%
% This template originates from:
% http://www.LaTeXTemplates.com
%
% Authors:
% Vel (vel@LaTeXTemplates.com)
% Alessandro Plasmati
%
% License:
% CC BY-NC-SA 3.0 (http://creativecommons.org/licenses/by-nc-sa/3.0/)
% 
%%%%%%%%%%%%%%%%%%%%%%%%%%%%%%%%%%%%%%%%%

%----------------------------------------------------------------------------------------
%	PACKAGES AND OTHER DOCUMENT CONFIGURATIONS
%----------------------------------------------------------------------------------------

\usepackage{etoolbox} % Required for conditional statements

\setlength\parindent{0pt} % Stop paragraph indentation

\usepackage{supertabular} % Required for the supertabular environment which allows tables to span multiple pages so sections with tables correctly wrap across pages

\usepackage{array} % for custom alignment if supertabular columns

\usepackage{ifthen} % For creating conditional entries

\usepackage{enumitem} % for formatting bulleted lists and reducing whitespace before bullet list

%----------------------------------------------------------------------------------------
%	DOCUMENT MARGINS
%----------------------------------------------------------------------------------------

\usepackage{geometry} % Required for adjusting page dimensions and margins

\geometry{
	hmargin=1.5cm, % Horizontal margin
	vmargin=1.75cm, % Vertical margin
	letterpaper, % Paper size, change to letterpaper for US letter size
	%showframe, % Uncomment to show how the type block is set on the page -- typically for debugging
}

%----------------------------------------------------------------------------------------
%	COLUMN LAYOUT
%----------------------------------------------------------------------------------------

\usepackage{paracol} % Required for creating multi-column layouts that can span pages automatically

\columnratio{0.55,0.45} % The relative ratios of the two columns in the CV

\setlength\columnsep{0.05\textwidth} % Specify the amount of space between the columns

%----------------------------------------------------------------------------------------
%	FONTS
%----------------------------------------------------------------------------------------

\usepackage{fontspec} % Required for specifying custom fonts under XeLaTeX

\setmainfont{EBGaramond}[ % Make the default font EBGaramond
Path=fonts/, % The font is provided with the template in the fonts folder
UprightFont=*-Regular.ttf,
BoldFont=*-Bold.ttf,
BoldItalicFont=*-BoldItalic.ttf,
ItalicFont=*-Italic.ttf,
SmallCapsFont=*-SC.ttf
]

\newfontfamily\cvtextfont[Path=fonts/]{freebooterscript} % Create a new font family for the cursive font Freebooter Script, provided with the template in the fonts folder

\newfontfamily\cvlangfont{Courier New} % Create a new font family for programming languages

\newfontfamily{\FA}[Path=fonts/]{FontAwesome} % Create a new font family for FontAwesome icons, provided with the template in the fonts folder
\input{fonts/fontawesomesymbols-xeluatex.tex} % Load a file to create shortcuts to the icons, see icon examples and their shortcuts in fontawesome.pdf in the fonts folder

\usepackage[sf,scale=0.95]{libertine} % Load Libertine as a \sffamily font for sans serif titles

%----------------------------------------------------------------------------------------
%	COLOURS AND LINKS
%----------------------------------------------------------------------------------------

\usepackage[usenames,svgnames]{xcolor} % Allows the definition and use of custom colours

\definecolor{text}{HTML}{2b2b2b} % Main document font colour, off-black
\definecolor{headings}{HTML}{701112} % Dark red colour for headings
\definecolor{shade}{HTML}{F5DD9D} % Peach colour for the contact information box
\definecolor{linkcolor}{HTML}{0000ff} % 25% desaturated headings colour for links
\definecolor{python}{HTML}{0E5484}
\definecolor{maingray}{HTML}{B9B9B9}
\definecolor{proglang}{HTML}{f2f2f2}
\definecolor{matlab}{HTML}{e16737}
\definecolor{cuda}{HTML}{3A4E3A}
\definecolor{rpackage}{HTML}{198CE7}
\definecolor{qt}{HTML}{3D6117}
% Other colour options: shade=B9D7D9 and linkcolor=A40000; shade=D4D7FE and linkcolor=FF0080

% For preset colours that can be used by their names without having to define them, see: https://www.latextemplates.com/svgnames-colors

\color{text} % Set the default text colour for the whole document to the colour defined as 'text' above

%------------------------------------------------

\usepackage{hyperref} % Required for links

\hypersetup{colorlinks, breaklinks, urlcolor=linkcolor, linkcolor=linkcolor} % Set up links and their colours


%----------------------------------------------------------------------------------------
%	HEADERS & FOOTERS
%----------------------------------------------------------------------------------------

\usepackage{fancyhdr} % Required for customising headers and footers

\pagestyle{fancy} % Enable custom headers and footers

\fancyhf{} % This suppresses all headers and footers by default, add headers and footers in the template file as per the example

\renewcommand{\headrulewidth}{0pt} % Remove the default rule under the header

%----------------------------------------------------------------------------------------
%	SECTIONS
%----------------------------------------------------------------------------------------

\usepackage[nobottomtitles*]{titlesec} % Required for modifying sections, the nobottomtitles* is required for pushing section titles to the next page when they are close to the bottom of the page

\renewcommand{\bottomtitlespace}{0.1\textheight} % Modify the minimal space required from the bottom margin not to move the title to the next page

\titleformat{\section}{\color{headings}\scshape\LARGE\raggedright}{}{0em}{}[\color{black}\titlerule] % Define the \section format

\titlespacing{\section}{0pt}{0pt}{8pt} % Spacing around section titles, the order is: left, before and after

%----------------------------------------------------------------------------------------
%	GRAPHICS DEFINITIONS
%---------------------------------------------------------------------------------------- 

\usepackage{tikz} % Required for creating the plots
\usetikzlibrary{shapes, backgrounds}
\tikzset{x=1cm, y=1cm} % Default tikz units

% Command to vertically centre adjacent content
\newcommand{\vcenteredhbox}[1]{% The only parameter is for the content to centre
	\begingroup%
		\setbox0=\hbox{#1}\parbox{\wd0}{\box0}%
	\endgroup%
}

%----------------------------------------------------------------------------------------
%	CHARTS
%---------------------------------------------------------------------------------------- 

\newcounter{barcount}

% Environment to hold a new bar chart
\newenvironment{barchart}[1]{ % The only parameter is the maximum bar width, in cm
	\newcommand{\barwidth}{0.35}
	\newcommand{\barsep}{0.2}
	
	% Command to add a bar to the bar chart
	\newcommand{\baritem}[2]{ % The first argument is the bar label and the second is the percentage the current bar should take up of the total width
		\pgfmathparse{##2}
		\let\perc\pgfmathresult
		
		\pgfmathparse{#1}
		\let\barsize\pgfmathresult
		
		\pgfmathparse{\barsize*##2/100}
		\let\barone\pgfmathresult
		
		\pgfmathparse{(\barwidth*\thebarcount)+(\barsep*\thebarcount)}
		\let\barx\pgfmathresult
		
		\filldraw[fill=shade, draw=none] (0,-\barx) rectangle (\barone,-\barx-\barwidth);
		
		\node [label=180:\colorbox{white}{\textcolor{text}{##1}}] at (0,-\barx-0.175) {};
		\addtocounter{barcount}{1}
	}
	\begin{tikzpicture}
		\setcounter{barcount}{0}
}{
	\end{tikzpicture}
}

%----------------------------------------------------------------------------------------
%	CUSTOM COMMANDS
%----------------------------------------------------------------------------------------

% Command for entering a new work position
\newcommand{\workposition}[7]{
	\multicolumn{2}{c}{
		\expandafter\ifstrequal\expandafter{#3}{}{}{\textbf{#3}} % Employer
		\expandafter\ifstrequal\expandafter{#4}{}{}{\hfill {\raggedright\textsc{#4}}} % Location
	}
	\expandafter\ifstrequal\expandafter{#3#4}{}{\\[-10pt]}{\\}
	\textsc{
		\expandafter\ifstrequal\expandafter{#1}{}{}{#1}
		\expandafter\ifstrequal\expandafter{#2}{}{}{\hspace{6pt}\footnotesize{(#2)}}
	} % Duration and conditional full time/part time text
	\expandafter\ifstrequal\expandafter{#5}{}{}{& {\textit{{#5}}}\\[4pt]} % Job title
	\expandafter\ifstrequal\expandafter{#6}{}{}{& #6} % Description
	\expandafter\ifstrequal\expandafter{#7}{}{}{\\[6pt]}
}

% Command for entering a separate qualification
\newcommand{\educationentry}[5]{
	\textsc{#1} & \textbf{#2} % Duration and degree
	\expandafter\ifstrequal\expandafter{#5}{}{}{& {\raggedright\textit{#5}}\\} % Institution
	\expandafter\ifstrequal\expandafter{#4}{}{}{& #4} % Department
	\expandafter\ifstrequal\expandafter{#3}{}{}{, {\small\textsc{#3}}\\} % Honours, achievements or distinctions (e.g. first class honours)
}

% Command for entering a separate table row -- used as a generic visual element for any section that uses a two column table
\newcommand{\tableentry}[3]{
	\textsc{#1} % for bullets
	& #2 % main text
	\expandafter\ifstrequal\expandafter{#3}{}{\\}{\\[6pt]} % First the heading, then content, then a conditional insertion of whitespace if the third parameter has any content in it
}

% Command for entering a separate table row -- used as a generic visual element for any section that uses a two column table
\newcommand{\tableentryraw}[3]{
	#1 & #2\expandafter\ifstrequal\expandafter{#3}{}{\\}{\\[6pt]} % First the heading, then content, then a conditional insertion of whitespace if the third parameter has any content in it
}

% Command for entering a separate table row -- used as a generic visual element for any section that uses a two column table
\newcommand{\tableentrythree}[4]{
	\textsc{#1} & #2 & \textsc{#3}\expandafter\ifstrequal\expandafter{#4}{}{\\}{\\[6pt]} % First the heading, then content, then the location, then a conditional insertion of whitespace if the third parameter has any content in it
}

% Command for entering a separate table row -- used as a generic visual element for any section that uses a two column table
\newcommand{\tableentryrawthree}[4]{
	\textsc{#1} & #2 & #3\expandafter\ifstrequal\expandafter{#4}{}{\\}{\\[6pt]} % First the heading, then content, then the location, then a conditional insertion of whitespace if the third parameter has any content in it
}


% Command for entering a long-form description where there is a title on one line and a paragraph description below it
\newcommand{\longformdescription}[2]{
	\textit{#1}\\[3pt]
	#2\medskip
}

% Command for entering a long-form description where there is a title on one line and a paragraph description below it
\newcommand{\longformdescriptiontight}[2]{
	\textit{#1}
	#2\medskip
}

% Command for entering a publication in long-form format
\newcommand{\longformpublication}[1]{
	#1\medskip
}

% Command for entering a publication as a short DOI (digital object identifier) string to the publication, a link is automatically created
\newcommand{\doipublication}[4]{
	#1 & % Year
	\href{http://dx.doi.org/#2}{\expandafter\ifstrequal\expandafter{#3}{firstauthor}{\textbf{doi:#2}}{doi:#2}}% DOI string and if "firstauthor" is entered for the 3rd argument, this makes the DOI string bold indicating a first author publication
	\expandafter\ifstrequal\expandafter{#4}{}{\\}{\\[3pt]} % Conditional insertion of whitespace if the 4th parameter has any content in it
}

% Command for creating skill level plots
\newcommand{\skilllevel}[2]{%
    \textcolor{black}{\textbf{#1}}~~~~~~\hfill
    \foreach \x in {1,...,10}{%
      \space{\ifnumgreater{\x}{#2}{\color{maingray}\faCircleO}{\color{headings}\faDotCircleO}}}\par%
}

% Define custom commands for CV info
\newcommand{\cvdate}[1]{\renewcommand{\cvdate}{#1}}
\newcommand{\cvmail}[1]{\renewcommand{\cvmail}{#1}}
\newcommand{\cvnumberphone}[1]{\renewcommand{\cvnumberphone}{#1}}
\newcommand{\cvaddress}[1]{\renewcommand{\cvaddress}{#1}}
\newcommand{\cvsite}[1]{\renewcommand{\cvsite}{#1}}
\newcommand{\aboutme}[1]{\renewcommand{\aboutme}{#1}}
\newcommand{\education}[1]{\renewcommand{\education}{#1}}
\newcommand{\references}[1]{\renewcommand{\references}{#1}}
\newcommand{\awards}[1]{\renewcommand{\awards}{#1}}
\newcommand{\profilepic}[1]{\renewcommand{\profilepic}{#1}}
\newcommand{\cvname}[1]{\renewcommand{\cvname}{#1}}
\newcommand{\cvjobtitle}[1]{\renewcommand{\cvjobtitle}{#1}}
\newcommand{\cvgithub}[1]{\renewcommand{\cvgithub}{#1}}
\newcommand{\cvlinkedin}[1]{\renewcommand{\cvlinkedin}{#1}}
\newcommand{\cvdescribe}[1]{\renewcommand{\cvdescribe}{#1}}
\newcommand{\cvskills}[1]{\renewcommand{\cvskills}{#1}}
\newcommand{\cvhobbies}[1]{\renewcommand{\cvhobbies}{#1}}
\newcommand{\cvreferences}[1]{\renewcommand{\cvreferences}{#1}}
\newcommand{\cvcourses}[1]{\renewcommand{\cvcourses}{#1}}

% Command to create the rounded boxes around the first three letters of section titles
\newcommand*\round[2]{%
	\begin{tikzpicture}[baseline=(char.base)]
		\node[align=center,anchor=north west, draw,rectangle, rounded corners=2.5mm, inner sep=1.6pt, minimum size=5.5mm, text centered, minimum height = 5mm, text height=2mm, fill=#2,#2,text=black](char){#1};%
	\end{tikzpicture}
}

% Command for creating a big colored dot
\newcommand{\programlang}[2]{
	\round{
		{#1$\bullet$}~~{\footnotesize\cvlangfont#2}
	}{proglang}
} % Include the file that specifies the document structure

% Headers and footers can be added with the \lhead{} \rhead{} \lfoot{} \rfoot{} commands
% Example right footer:
%\rfoot{\color{headings}{\sffamily Last update: \today. Typeset with Xe\LaTeX}}

%----------------------------------------------------------------------------------------
%	 PERSONAL INFORMATION
%----------------------------------------------------------------------------------------

% If you don't need one or more of the below, just remove the content leaving the command, e.g. \cvnumberphone{}

\profilepic{} % Profile picture

\cvname{Khalil Al Handawi, PhD} % Your name
\cvjobtitle{Engineer, designer, and researcher} % Job title/career

\cvdate{} % Date of birth
\cvaddress{Montr\'{e}al Qu\'{e}bec, Canada} % Short address/location, use \newline if more than 1 line is required
\cvnumberphone{} % Phone number
\cvsite{khbalhandawi.github.io} % Personal website
\cvmail{khalil.alhandawi@mail.mcgill.ca} % Email address
\cvgithub{github.com/khbalhandawi} % GitHub
\cvlinkedin{linkedin.com/in/khbalhandawi} % LinkedIn

\aboutme{} % To have no About Me section, just remove all the text and leave \aboutme{}

\begin{document}

\begin{paracol}{2} % Begin the multi-column environment

%----------------------------------------------------------------------------------------
%	NAME AND CURRICULUM VITAE TEXT
%----------------------------------------------------------------------------------------

\parbox[top][0.1\textheight][c]{\linewidth}{ % Parbox to hold the author name and CV text; fixed height to match the coloured box to the right, centred vertically and full line width
	\vspace{-0.00\textheight} % Reduce whitespace above the parbox to separate it from the main content
	\centering % Centre text
	{\sffamily\Huge \cvname}\\\medskip % Your name
}

%----------------------------------------------------------------------------------------

\switchcolumn % Switch to the next paracol column

%----------------------------------------------------------------------------------------
%	COLOURED CONTACT DETAILS BOX
%----------------------------------------------------------------------------------------

\parbox[top][0.1\textheight][c]{\linewidth}{ % Parbox to hold the colour box; fixed height to match the name/CV text to the left, centred vertically and full line width
	\vspace{0\textheight} % Reduce whitespace above the parbox to separate it from the main content
	\colorbox{shade}{ % Create the coloured box
		\begin{supertabular}{p{0.05\linewidth}|p{0.775\linewidth}} % Start a table with two columns, the table will ensure everything is aligned
			\ifthenelse{\equal{\cvdate}{}}{}{\raisebox{-1pt}{\faInfo} & \cvdate \\}
			\ifthenelse{\equal{\cvaddress}{}}{}{\raisebox{-1pt}{\faHome} & \cvaddress \\} % Address
			\ifthenelse{\equal{\cvnumberphone}{}}{}{\raisebox{-1pt}{\faPhone} & \cvnumberphone \\} % Phone number
			\ifthenelse{\equal{\cvmail}{}}{}{\raisebox{0pt}{\small\faEnvelope} & \href{mailto:\cvmail}{\cvmail} \\} % Email address
			\ifthenelse{\equal{\cvsite}{}}{}{\raisebox{-1pt}{\small\faGlobe} & \href{https://\cvsite}{\cvsite} \\} % Website
			\ifthenelse{\equal{\cvgithub}{}}{}{\raisebox{-1pt}{\faGithub} & \href{https://\cvgithub}{\cvgithub} \\} % GitHub profile
			\ifthenelse{\equal{\cvlinkedin}{}}{}{\raisebox{-1pt}{\faLinkedinSquare} & \href{https://\cvlinkedin}{\cvlinkedin} \\} % LinkedIn profile
			% See fontawesome.pdf in the fonts folder for all icons you can use
		\end{supertabular}
	}
}

\end{paracol}

%----------------------------------------------------------------------------------------
%	ABOUT ME
%----------------------------------------------------------------------------------------

\ifthenelse{\equal{\aboutme}{}}{}{
	\section{About me}
	\begin{raggedright}
		\aboutme
	\end{raggedright}
}

%----------------------------------------------------------------------------------------
%	EDUCATION
%----------------------------------------------------------------------------------------

\section{Education} 

% Blank \educationentry{} command to add another degree:

%\educationentry{} % Duration
%{} % Degree
%{} % Honours, achievements or distinctions (e.g. first class honours)
%{} % Department
%{} % Institution

% All 5 parameters must be supplied but any can be empty if you don't need them

%------------------------------------------------

\begin{supertabular}{>{\raggedleft\arraybackslash}p{0.17\linewidth}p{0.6\linewidth}>{\raggedleft\arraybackslash}p{0.17\linewidth}} % Start a table with three columns, the table will ensure everything is aligned

	%------------------------------------------------

	\educationentry{Jan 2017 -- Dec 2020} % Duration
	{Doctor of Philosophy} % Degree
	{CGPA: 4.00} % Honours, achievements or distinctions (e.g. first class honours)
	{Mechanical Engineering} % Department
	{McGill University} % Institution
	\tableentrythree{Concentraion}{Engineering design and optimization}{}{}
	\tableentrythree{Dissertation}{\textit{Optimization driven set-based design under uncertain requirements}}{}{spaceafter}
	
	%------------------------------------------------

	\educationentry{Aug 2013 -- Dec 2015} % Duration
	{Master of Science} % Degree
	{CGPA: 4.00} % Honours, achievements or distinctions (e.g. first class honours)
	{Mechanical Engineering} % Department
	{Khalifa University} % Institution
	\tableentrythree{Concentraion}{Instrumentation and photonics}{}{}
	\tableentrythree{Dissertation}{\textit{Internal corrosion detection of oil and gas pipelines using fiber optics}}{}{spaceafter}
	
	%------------------------------------------------

	\educationentry{Aug 2009 -- June 2013} % Duration
	{Bachelor of Science} % Degree
	{First Class Honours, CGPA: 3.97} % Honours, achievements or distinctions (e.g. first class honours)
	{Mechanical Engineering} % Department
	{Khalifa University} % Institution
	\tableentrythree{Capstone project}{\textit{Development of a human operated mobile hexapod platform}}{}{}
	%------------------------------------------------

\end{supertabular}

%----------------------------------------------------------------------------------------
%	ACADEMIC EXPERIENCE
%----------------------------------------------------------------------------------------

\section{Work Experience}

% Blank \workposition command to add another job:

%\workposition{} % Duration
%{} % FT/PT (full time or part time)
%{} % Employer
%{} % Location
%{} % Job title
%{} % Description
%{} % Space after

% All 7 parameters must be supplied but any can be empty if you don't need them

\textbf{\large Research}\medskip

%------------------------------------------------

% \begin{supertabular}{>{\raggedleft\arraybackslash}p{0.17\linewidth}p{0.6\linewidth}>{\raggedleft\arraybackslash}p{0.17\linewidth}} % Start a table with three columns, the table will ensure everything is aligned
\begin{supertabular}{>{\raggedleft\arraybackslash}p{0.17\linewidth}p{0.79\linewidth}} % Start a table with three columns, the table will ensure everything is aligned

	%------------------------------------------------

	\workposition{May 2022 -- present} % Duration
	{} % FT/PT (full time or part time)
	{Department of Computer Science and Operations Research, Universit\'{e} de Montr\'{e}al} % Employer
	{Montr\'{e}al, Canada} % Location
	{Postdoctoral Researcher} % Job title
	{}{} % Description
	\tableentry{$\bullet$}{Developed a research plan and won a Natural Sciences and Engineering Research Council of Canada (NSERC) fellowship to pursue said research at Universite de Montr\'{e}al.}{}
	\tableentry{$\bullet$}{Work with the International Air Transport Association (IATA) to develop data analytics solutions for codesharing and flight scheduling in the civil aviation industry.}{}
	\tableentry{$\bullet$}{Assess the effectiveness and impact of the IATA operation safety audit (IOSA) on air travel accessibility and cooperation between airlines.}{spaceafter}

	%------------------------------------------------

	\workposition{Jan 2021 -- Apr 2022} % Duration
	{} % FT/PT (full time or part time)
	{Systems Optimization Lab (SOL), McGill University} % Employer
	{Montr\'{e}al, Canada} % Location
	{Postdoctoral Researcher} % Job title
	{}{} % Description
	\tableentry{$\bullet$}{Researched simulation-based decision-making for public health and policy-making during epidemics.}{}
	\tableentry{$\bullet$}{Developed several \href{https://covid-forecaster-lebanon.herokuapp.com/}{machine learning COVID-19 forecasting models} for infering weekly pandemic trajectories.}{}
	\tableentry{$\bullet$}{Applied state-of-the-art stochastic black-box optimization algorithms to tune machine learning hyperparameters and model selection.}{}
	\tableentry{$\bullet$}{Develop GPU accelerated \href{https://github.com/khbalhandawi/COVID_SIM_GPU}{epidemic models} for high simulation throughput to assist in policy-making during pandemics through the use of stochastic optimization.}{}
	\tableentry{$\bullet$}{Invited to speak at the \href{https://meetings.informs.org/wordpress/healthcare2023/}{INFORMS Healthcare conference 2023} in Toronto.}{}

	%------------------------------------------------

	\workposition{Jan 2017 -- Jan 2021} % Duration
	{} % FT/PT (full time or part time)
	{} % Employer
	{} % Location
	{Research assistant} % Job title
	{}{}  % Description
	\tableentry{$\bullet$}{I was part of \textbf{Canadian/European industrial project} investigating additive repair technologies for aeroengine parts. I focused on optimization of aerospace design for AM remanufacturing.}{}
	\tableentry{$\bullet$}{Developed mathematical tools and \href{https://github.com/khbalhandawi/scale_AM_webapp}{software} for design space exploration and optimization achieving a \textbf{99.8\% reduction} in effort to explore a 4-dimensional design space relative to full factorial design.}{}
	\tableentry{$\bullet$}{Developed a thermomechanical simulation model for modeling additive manufacturing repair and life extension processes using transient coupled thermal/mechanical FEA simulations.}{}
	\tableentry{$\bullet$}{Used machine learning models trained from expensive manufacturing simulations and a variant of kernel smoothing for estimating the sensitivity of design and process parameters to different requirements.}{}
	\tableentry{$\bullet$}{My doctoral research paper was selected as a \textbf{\href{https://www.gknaerospace.com/en/our-technology/2022/additive-manufacturing-research-paper-receives-journal-of-mechanical-design-award/}{best paper}} by the ASME Journal of Mechanical Design in 2021.}{}
	\tableentry{$\bullet$}{Recived a Fonds de Recherche du Qu\'{e}bec (FRQNT) doctoral award.}{}
	\tableentry{$\bullet$}{Co-developed a novel lifecycle cost model based on system dynamics to model the effect of life extension on lifecycle costs.}{spaceafter}

	%------------------------------------------------

	\workposition{Sep 2021 -- Dec 2021} % Duration
	{} % FT/PT (full time or part time)
	{Systems Engineering Design (SED) Lab, Chalmers University of Technology} % Employer
	{G\"{o}teborg, Sweden} % Location
	{Postdoctoral Researcher} % Job title
	{}{} % Description
	\tableentry{$\bullet$}{Integrate my doctoral research (design under uncertainty) into SED lab activities.}{}
	\tableentry{$\bullet$}{Research change propagation and absorption in engineering design (applied to aeroengine systems).}{}
	\tableentry{$\bullet$}{Authored a \href{https://sed-group.github.io/mvmlib/index.html}{Python library} for margin and change propagation management in engineering systems.}{}
	\tableentry{$\bullet$}{Used said library in design space exploration to concurrently develop and analyze \textbf{6,552 conceptual designs} of an aeroengine component and \href{https://khbalhandawi.github.io/projects/1_project/}{visualize the results} using interactive tools.}{spaceafter}

	%------------------------------------------------

	\workposition{Aug 2013 -- Jan 2017} % Duration
	{} % FT/PT (full time or part time)
	{Asset Integrity Management Systems (AIMS) Lab, Khalifa University} % Employer
	{Abu Dhabi, UAE} % Location
	{Research Assistant} % Job title
	{}{} % Description
	\tableentry{$\bullet$}{Developed fiber optic structural monitoring sensors for mitigating upto \textbf{\$1M of corrosion costs}.}{}
	\tableentry{$\bullet$}{Developed and simulated a \href{https://khbalhandawi.github.io/projects/6_project/}{fiber optic-based corrosion sensor} with an accuracy of \textbf{$\pm$2mm/s}.}{}
	\tableentry{$\bullet$}{Developed a new accelerated corrosion testing setup to simulate \textbf{2 years of corrosion in 2 hours}.}{spaceafter}

	%------------------------------------------------

\end{supertabular}

%------------------------------------------------

% \vspace{-\baselineskip}\medskip % Standardise the whitespace after this section and before the next (the custom command adds too much otherwise)

%----------------------------------------------------------------------------------------
%	WORK EXPERIENCE
%----------------------------------------------------------------------------------------

\hrule \medskip \textbf{\large Industry}\medskip

%------------------------------------------------

\begin{supertabular}{>{\raggedleft\arraybackslash}p{0.17\linewidth}p{0.79\linewidth}} % Start a table with three columns, the table will ensure everything is aligned

	%------------------------------------------------

	\workposition{May 2022 -- present} % Duration
	{} % FT/PT (full time or part time)
	{International Air Transport Association (IATA)} % Employer
	{Montr\'{e}al, Canada} % Location
	{Visiting researcher} % Job title
	{}{}  % Description
	\tableentry{$\bullet$}{Analyze IATA data involving \textbf{25M flight schedules} using graph representation learning.}{}
	\tableentry{$\bullet$}{Develop community detection algorithms for graphs with over \textbf{10K nodes} and \textbf{100K edges}.}{}
	\tableentry{$\bullet$}{Organize flight schedule data into a mySQL database for  archival, and retrieval of data.}{spaceafter}

	%------------------------------------------------

	\workposition{June 2017 -- Jan 2020} % Duration
	{} % FT/PT (full time or part time)
	{GKN Aerospace Engine Systems} % Employer
	{Trolh\"{a}ttan, Sweden} % Location
	{Visiting researcher} % Job title
	{}{}  % Description
	\tableentry{$\bullet$}{Participated in a \href{https://khbalhandawi.github.io/projects/4_project/}{technology transfer} at GKN Aerospace to translate my doctoral research into industrial practice by provided training modules and workshops to GKN engineers on the software tools that I have developed.}{}
	\tableentry{$\bullet$}{Surveyed GKN engineers about their experience designing areoengine components for engine system manufacturers to create a timeline of expected design updates and changes. This data formed the basis of a case study for my research on design for flexibility and robustness.}{}
	\tableentry{$\bullet$}{Set up advanced design automation and exploration tools to be used as part of GKN's workflow (engineering workbench) by integrated parametric design software (NX Siemens) with simulation software (Abaqus and ANSYS) to evaluate hundreds of concepts for a turbine rear frame component.}{spaceafter}

	%------------------------------------------------

	\workposition{Aug 2012 -- May 2012} % Duration
	{} % FT/PT (full time or part time)
	{Yokogawa} % Employer
	{Abu Dhabi, UAE} % Location
	{Engineering intern} % Job title
	{}{} % Description
	\tableentry{$\bullet$}{Created software and programs for industrial plant operation and control using Distributed Control Systems.}{}
	\tableentry{$\bullet$}{Visited the main headquarters in Japan to represent the Abu Dhabi National Oil Company.}{spaceafter}

	%------------------------------------------------

\end{supertabular}

%------------------------------------------------

\vspace{-\baselineskip}\medskip % Standardise the whitespace after this section and before the next (the custom command adds too much otherwise)

%----------------------------------------------------------------------------------------
%	MANAGEMENT EXPERIENCE
%----------------------------------------------------------------------------------------

\hrule \medskip \textbf{\large Management}\medskip 

%------------------------------------------------

\begin{supertabular}{>{\raggedleft\arraybackslash}p{0.17\linewidth}p{0.79\linewidth}} % Start a table with three columns, the table will ensure everything is aligned

	%------------------------------------------------

	% \workposition{Jan 2021 -- present} % Duration
	% {} % FT/PT (full time or part time)
	% {BPGIC holdings Limited} % Employer
	% {Dubai, UAE} % Location
	% {Non-executive independent member of board of directors} % Job title
	% {}{} % Description
	% \tableentry{$\bullet$}{Make decisions on the BPGIC holdings Limited board of directors to further the company's objective of expanding its operations in the energy sector.}{}
	% \tableentry{$\bullet$}{Attend quarterly board of directors meetings and provide expert opinion and advice.}{spaceafter}

	%------------------------------------------------

	\workposition{Jan 2017 -- Dec 2020} % Duration
	{} % FT/PT (full time or part time)
	{McGill University} % Employer
	{Montr\'{e}al, Canada} % Location
	{Systems Optimization Lab} % Job title
	{}{} % Description
	\tableentry{$\bullet$}{Update the lab's website and disseminate new research to the public.}{}
	\tableentry{$\bullet$}{\href{http://www.sol.research.mcgill.ca/}{http://www.sol.research.mcgill.ca/}.}{spaceafter}

	%------------------------------------------------

	\workposition{Jan 2014 -- Aug 2015} % Duration
	{} % FT/PT (full time or part time)
	{Khalifa University} % Employer
	{Abu Dhabi, UAE} % Location
	{Solar car project: Maintenance and procurement manager} % Job title
	{}{} % Description
	\tableentry{$\bullet$}{Designed an engineering workshop for building and maintaining electric solar vehicles.}{}

	%------------------------------------------------

	\workposition{Jun 2011 -- May 2013} % Duration
	{} % FT/PT (full time or part time)
	{} % Employer
	{} % Location
	{Baja SAE team leader} % Job title
	{}{} % Description
	\tableentry{$\bullet$}{Was elected to lead the team during the Baja SAE 2013 and 2015 international competitions.}{}
	\tableentry{$\bullet$}{Saw the project to completion and was recognized for leading the first UAE national team to participate in the Baja SAE competition.}{spaceafter}

	%------------------------------------------------

\end{supertabular}

%------------------------------------------------

\vspace{-\baselineskip}\medskip % Standardise the whitespace after this section and before the next (the custom command adds too much otherwise)

%----------------------------------------------------------------------------------------
%	TEACHING EXPERIENCE
%----------------------------------------------------------------------------------------

\section{Teaching and supervision}

\textbf{\large Teaching} \medskip 

%------------------------------------------------

\begin{supertabular}{>{\raggedleft\arraybackslash}p{0.17\linewidth}p{0.79\linewidth}} % Start a table with three columns, the table will ensure everything is aligned

	\workposition{Sep 2022 -- Dec 2022} % Duration
	{} % FT/PT (full time or part time)
	{McGill University} % Employer
	{Montr\'{e}al, Canada} % Location
	{Adjunct lecturer, MECH559: Engineering Systems Optimization} % Job title
	{}{} % Description
	\tableentry{$\bullet$}{Developed \href{https://github.com/khbalhandawi/MECH559_notebooks/tree/master}{Python notebooks} as teaching aids for the students to understand the implementation of common optimization algorithms and their application to engineering systems.}{}
	\tableentry{$\bullet$}{Received participation rate of \textbf{69\%} in course evaluations with an average score of \textbf{4.2/5.0}.}{}
	\tableentry{$\bullet$}{Hosted \textbf{several guest} lectures with aerospace industry professionals to demo optimization applications.}{spaceafter}

	%------------------------------------------------

	\workposition{Sep 2018 -- Dec 2019} % Duration
	{} % FT/PT (full time or part time)
	{} % Employer
	{} % Location
	{Teaching assistant, MECH362: Mechanical Lab} % Job title
	{}{} % Description
	\tableentry{$\bullet$}{Was a teaching assistant for MECH362 for 3 semesters.}{}
	\tableentry{$\bullet$}{Supervised lab sessions covering the following theoretical courses: MECH240 Thermodynamics, MECH315/419 Mechanics, MECH331 Fluid Mechanics, MECH346 Heat Transfer.}{}
	\tableentry{$\bullet$}{Prepared lab manuals, conducted labs, graded student reports, and provided feedback.}{}
	\tableentry{$\bullet$}{Reported areas of the curriculum that students were constantly struggling with to the department.}{}

	%------------------------------------------------

	\workposition{Jan 2018 -- May 2018} % Duration
	{} % FT/PT (full time or part time)
	{} % Employer
	{} % Location
	{Teaching assistant, FACC400: Engineering Professional Practice} % Job title
	{}{} % Description
	\tableentry{$\bullet$}{Was a teaching assistant for FACC400 for 1 semester.}{}
	\tableentry{$\bullet$}{Conducted town halls on relevant and pressing sociapolitical topics facing modern day engineers.}{}
	\tableentry{$\bullet$}{Substituted lecturers, and provided feedback to students on the townhalls.}{}

	%------------------------------------------------

	\workposition{Jan 2014 -- May 2014} % Duration
	{} % FT/PT (full time or part time)
	{Khalifa University} % Employer
	{Abu Dhabi, UAE} % Location
	{Teaching assistant, System Dynamics and Controls} % Job title
	{}{} % Description
	\tableentry{$\bullet$}{Conducted lab sessions on transient systems and multiple degree of freedom systems.}{}
	\tableentry{$\bullet$}{Prepared homework solutions and tutorials for the students on theoretical topics.}{}
	\tableentry{$\bullet$}{Graded students' midterms examinations.}{}

	%------------------------------------------------

	% \tableentrythree{Sep 2013 – Dec 2013}{Teaching assistant for the computer aided design course. Conducted computer lab sessions where students were taught CAD basics and guidelines for producing professional engineering drawings.}{Khalifa University, Abu Dhabi, UAE}{spaceafter}

	\workposition{Sep 2013 -- Dec 2013} % Duration
	{} % FT/PT (full time or part time)
	{} % Employer
	{} % Location
	{Teaching assistant, Computer Aided Design} % Job title
	{}{} % Description
	\tableentry{$\bullet$}{Conducted computer lab sessions to educate the students on the use of modern CAD software (SolidWorks).}{}

	%------------------------------------------------

	\workposition{Sep 2012 -- May 2013} % Duration
	{} % FT/PT (full time or part time)
	{} % Employer
	{} % Location
	{Grader, Physics II} % Job title
	{}{} % Description
	\tableentry{$\bullet$}{Graded homework, midterms, and final examinations for the undergraduate physics course.}{spaceafter}

\end{supertabular}

%------------------------------------------------

% \vspace{-\baselineskip}\medskip % Standardise the whitespace after this section and before the next (the custom command adds too much otherwise)

%----------------------------------------------------------------------------------------
%	STUDENT EXPERIENCE
%----------------------------------------------------------------------------------------

\hrule \medskip \textbf{\large Supervision} \medskip 

%------------------------------------------------

\begin{supertabular}{>{\raggedleft\arraybackslash}p{0.17\linewidth}p{0.6\linewidth}>{\raggedleft\arraybackslash}p{0.17\linewidth}} % Start a table with three columns, the table will ensure everything is aligned

	%------------------------------------------------

	\tableentrythree{Jan 2016 -- Jan 2017}{{\textbf{Asset Integrity Management Systems Lab (AIMS), Khalifa University}}}{Abu Dhabi, UAE}{spaceafter}
	\tableentry{$\bullet$}{Student name: Safieh Almahmoud (Masters student)}{} % Description
	\tableentry{}{Affiliation: Khalifa University}{}
	\tableentry{}{Domain: Solid mechanics, instrumentation, and photonics}{spaceafter}
	\tableentry{$\bullet$}{Student name: Tasneem Osman (Masters student)}{} % Description
	\tableentry{}{Affiliation: Khalifa University}{}
	\tableentry{}{Domain: Solid mechanics, instrumentation, and acoustics}{spaceafter}

	%------------------------------------------------
	\tableentrythree{Jan 2014 –- Aug 2015}{{\textbf{Capstone project, Khalifa University}}}{Abu Dhabi, UAE}{spaceafter}
	\tableentry{$\bullet$}{Student name: Yazan Hindawi (Bachelors student)}{} % Description
	\tableentry{}{Affiliation: Khalifa University}{}
	\tableentry{}{Domain: Solid mechanics, instrumentation, and robotics}{spaceafter}
	\tableentry{$\bullet$}{Student name: Ali Shamlan (Masters student)}{} % Description
	\tableentry{}{Affiliation: Khalifa University}{}
	\tableentry{}{Domain: Solid mechanics, instrumentation, and robotics}{spaceafter}
	
	%------------------------------------------------

\end{supertabular}

%------------------------------------------------

\vspace{-\baselineskip}\medskip % Standardise the whitespace after this section and before the next (the custom command adds too much otherwise)

%----------------------------------------------------------------------------------------
%	MAJOR RESEARCH PROJECT
%----------------------------------------------------------------------------------------

\section{Description of selected publications}

{\raggedright\textbf{``Scalable set-based design optimization and remanufacturing for meeting changing requirements"}}
\textit{\center How do you quantify the remanufacturability of a product before it goes into production?}\\

This question was motivated by the advent of novel manufacturing technologies such as additive manufacturing (AM) and their enormous potential to enable a circular economy recovery activities such as remanufacturing. In this paper, I highlight this potential and answer the above research question by using quantitative metrics to measure the design's remanufacturability when using additive or conventional subtractive manufacturing processes.

The metric was derived from the principle of design changeability, and the concept of scalability specifically. It was found that scalability is relevant to remanufacturing as it defines the potential for restoring product specifications to its original or better-than-original levels. I mathematically formulated a metric for scalability and used it in design space exploration (DSE) of an aeroengine component at GKN Aerospace engine systems to identify a set of scalable designs that are eligible for remanufacturing via additive manufacturing. The results show great promise and allow designers to incorporate the principles of sustainable manufacturing and design early in the product development cycle.

\programlang{\color{python}}{Python}
~\programlang{\color{matlab}}{MATLAB}
~\programlang{\color{black}}{NX Siemens}
~\programlang{\color{black}}{Abaqus}
~\programlang{\color{black}}{Thermomechanical simulation}
~\programlang{\color{black}}{Design space exploration (DSE)}
~\programlang{\color{black}}{Surrogate modeling}
~\round{\href{https://github.com/khbalhandawi/DM_SBD_opt}{\small\faGlobe~Open-source code}}{proglang}
~\round{\href{https://scale-am.herokuapp.com/}{\small\faGlobe~Web application}}{proglang}
~\round{\href{https://asmedigitalcollection.asme.org/mechanicaldesign/article-abstract/143/2/021702/1085767/Scalable-Set-Based-Design-Optimization-and}{\small\faGlobe~DOI: 10.1115/1.4047908}}{proglang}

{\raggedright\textbf{``Optimization of design margins allocation when making use of additive remanufacturing"}}
\textit{\center How do you design a component when the design requirements can change at any moment and without advance notice?}\\

That is the question my dissertation tries to answer. To answer this question, I needed to identify the mechanisms by which products are able to mitigate requirement changes. The literature suggested passive methods such as the use of design margins and active methods such as changing the product's design. Each method has its advantages and disadvantages and balancing the two strategies within a product is key to cost-effective mitigation of changing requirements. I assessed different aeroengine product designs (from GKN aerospace) against a wide variety of requirement change scenarios (using Monte Carlo simulation) to identify those designs that absorbed the most change without negatively impacting the product's performance (in terms of weight and redesign cost). The results of the study show promise and the open-source design tool that was developed allows designers to conceive of lean products despite the uncertain design requirements they have to work with.

\programlang{\color{python}}{Python}
~\programlang{\color{magenta}}{C++}
~\programlang{\color{matlab}}{MATLAB}
~\programlang{\color{rpackage}}{R}
~\programlang{\color{black}}{NX Siemens}
~\programlang{\color{black}}{Abaqus}
~\programlang{\color{black}}{Monte Carlo simulation}
~\programlang{\color{black}}{Uncertainty quantification}
~\round{\href{https://github.com/khbalhandawi/DM_SBD_opt}{\small\faGlobe~Online open-source code}}{proglang}
~\round{\href{https://asmedigitalcollection.asme.org/mechanicaldesign/article-abstract/144/1/012001/1112332/Optimization-of-Design-Margins-Allocation-When?redirectedFrom=fulltext}{\small\faGlobe~DOI: 10.1115/1.4051607}}{proglang}


\medskip % Extra whitespace before the next section

{\raggedright\textbf{``Optimization of Infectious Disease Prevention and Control Policies Using Artificial Life"}}
\textit{\center How can we apply the principles of design and decision making to help bring a pandemic under control?}\\

Although this project is not directly relevant to the discipline of materials science and industrial engineering, I found it quite useful for my design research. This is because most design problems involve a fair bit of uncertainty at all stages of the product development process. Being able to explore the design space under uncertainty is a very challenging problem mathematically. This project exemplifies such design problems by treating the public health policies for an epidemic as design solutions. 

I modeled how an infectious disease spreads in a small population. Diseases such as COVID-19 spread through social interaction. I programmed intelligent agents to model a complex social system. I used stochastic derivative-free optimization to determine the critical amount of intervention necessary to keep the disease in check without negatively affecting the economy. I used the stochastic optimization algorithm to reduce the number of hospitalizations beneath the healthcare capacity while reducing the socio-economic cost of interventions by up to \textbf{5 times} compared to a complete lock-down. Such tradeoffs are quite common in the engineering design world and I plan to use stochastic design exploration strategies in my future design research.

\programlang{\color{magenta}}{C++}
~\programlang{\color{cuda}}{CUDA}
~\programlang{\color{python}}{Python}
~\programlang{\color{qt}}{Qt}
~\programlang{\color{black}}{Stochastic optimization}
~\programlang{\color{black}}{Agent-based modeling}
~\round{\href{https://github.com/khbalhandawi/COVID_SIM}{\small\faGlobe~Online open-source code}}{proglang}
~\round{\href{https://ieeexplore.ieee.org/abstract/document/9532002}{\small\faGlobe~DOI: 10.1109/TETCI.2021.3107496}}{proglang}

\medskip % Extra whitespace before the next section

% {\raggedright\textbf{``A lifecycle cost-driven system dynamics
% approach for considering additive re-manufacturing or repair in aero-engine component design"}}
% \textit{\center How can we minimize the lifecycle cost when considering additve remanufactuing life extension strategies?}\\

% I collaborated with a colleague that developed a novel lifecycle cost (LCC) model based on system dynamics to capture the causal loops that often arise in LCC modeling. Such casual loops include viscous cycles, where increasing one parameter leads to another increasing which in turn results in the former parameter increasing even more. Such loops are difficult to capture in traditional time-driven activity-based LCC modeling and hence the need for a system dynamics model. The developed LCC model was used to explore different life extension strategies for aeroengine products from an LCC perspective. I formulated and solved an optimization problem that captures the tradeoff between life extension strategies that favor ``design for-life'' versus strategies that favor frequent life extension and maintenance. Me and my colleague managed to present a tool for obtaining the optimal life extension schedule such that LCC is minimized.

% \programlang{\color{matlab}}{MATLAB}
% ~\programlang{\color{black}}{Simulink}
% ~\programlang{\color{black}}{Lifecycle cost (LCC)}
% ~\programlang{\color{black}}{System dynamics modeling}
% ~\round{\href{https://www.cambridge.org/core/journals/proceedings-of-the-international-conference-on-engineering-design/article/lifecycle-costdriven-system-dynamics-approach-for-considering-additive-remanufacturing-or-repair-in-aeroengine-component-design/E6288AA6947FED87C8AD308E3CA62788}{\small\faGlobe~DOI: 10.1017/dsi.2019.140}}{proglang}

\medskip % Extra whitespace before the next section

%----------------------------------------------------------------------------------------
%	PREVIOUS RESEARCH PROJECT
%----------------------------------------------------------------------------------------

% {\raggedright\textbf{``Strain based FBG sensor for real-time corrosion rate monitoring in pre-stressed structures"}}\\
% Oil pipelines are monitored for corrosion on regular intervals using conventional tools. This activity accrues massive maintenance costs on the pipeline operator. I tried to mitigate maintenance costs by developing a passive realtime corrosion monitoring product. This research was my first exposure to the industrial world and the importance of developing cost-effective solutions. Furthermore, this was my first real product-design problem, where I employed the principles of product development to arrive at a prototype solution that the industry can readily test.
% I used concept elimination to find a solution that best fulfils the requirements of oil pipeline operators of being intrinsically safe and requiring minimal energy to operate. I arrived at a concept that uses fiber optics to transmit and receive signals from the pipeline without needed much energy to excite the laser and posing no electrical hazards to the pipeline environment. This sensor consists of fiber Bragg grating (FBG), which is used to sense the change in the pipeline's hoop stress as a result of internal corrosion and localized thinning of its inner diameter. The system was simulated, and a laboratory scale prototype was 3D printed to validate and test the solution. Our setup featured advanced fiber optic sensors (fiber Bragg grating (FBG)) and spectral analyzers controlled and operated by LabVIEW data acquisition software. I also relied heavily on MATLAB and Abaqus simulations to construct and verify the setup.

% \programlang{\color{black}}{LabVIEW}
% ~\programlang{\color{black}}{Abaqus}
% ~\programlang{\color{matlab}}{MATLAB}
% ~\programlang{\color{black}}{Fiber Bragg grating (FBG)}
% ~\round{\href{https://www.sciencedirect.com/science/article/abs/pii/S0925400516308528}{\small\faGlobe~DOI: 10.1016/j.snb.2016.05.167}}{proglang}

\medskip % Extra whitespace before the next section

% {\raggedright\textbf{``Development of a human operated mobile hexapod platform"}}\\
% I designed and developed of a six-legged hexapod platform that employed a hydraulic system to actuate each of the six robotic manipulators used for locomotion. The control and distribution of the hydraulic power was controlled through state-of-the art technology employing hydraulic servo-valves. As of today, the design stands at a total weight of 3 tons and occupies a space of 5m x 2m x 3m high. The robotic manipulator which constitutes a leg of the hexapod was reverse engineered and repurposed for walking from a used hydraulic excavator manufactured by JCB. I made a faithful reproduction of the 3D CAD model in SOLIDWORKS by taking measurements from the physical excavator so that the necessary modifications can be designed and manufactured. Most modifications were manufactured manually using a milling machine, lathe, and MIG/TIG welding. More complex parts where tolerances are critical were machined via a CNC milling machine. Once the mechanical setup was built, I used kinematics, and rotation matrices to compute the necessary actuation for tracking a given trajectory. I used and configured a microcontroller (Arduino) to provide the actuation signal to the servo-valves that distribute hydraulic power throughout the hexapod and achieved smooth tracking of the target trajectory that I provided.

% \programlang{\color{black}}{SOLIDWORKS}
% ~\programlang{\color{black}}{manufacturing}
% ~\programlang{\color{black}}{CNC machining}
% ~\programlang{\color{black}}{kinematics}
% ~\programlang{\color{black}}{rotation matrices}
% ~\programlang{\color{black}}{Arduino}

% \medskip % Extra whitespace before the next section

% %----------------------------------------------------------------------------------------
% %	EXTRACURRICULAR PROJECTS
% %----------------------------------------------------------------------------------------
% \newpage
% \section{Extracurricular projects}

% {\raggedright\textbf{``Khalifa university Solar Car Project"}}\\
% I was part of the Khalifa University Solar Car Team which successfully developed and commissioned the \textbf{first} solar vehicle in The United Arab Emirates. The vehicle was of outstanding performance and competed in The Abu Dhabi Solar Challenge (January 15th – 19th 2015) earning \textbf{second place} in the race with 2 minutes behind the leading team (University of Michigan, Ann Arbor). My responsibilities in the team included aerodynamic design of the body and driver canopy using ANSYS Fluent. Furthermore, after the design stage my main responsibilities were to oversee the manufacturing of the various components needed to operate the vehicle as well as procure the equipment needed to build and maintain the vehicle. In order to fabricate the monocoque carbon fiber chassis of the vehicle, I collaborated with a team of solar car designers at \textbf{Tokai University, Japan}. The project provided a lot of hands-on inter-disciplinary experience in instrumentation, mechanical, and electrical design.

% \programlang{\color{black}}{aerodynamic design}
% ~\programlang{\color{black}}{ANSYS Fluent}
% ~\programlang{\color{black}}{carbon fiber}
% ~\round{\href{https://www.thenationalnews.com/business/technology/solar-team-awaits-its-day-in-the-sun-1.601904}{\small\faGlobe
% ~Online news article}}{proglang}

% \medskip % Extra whitespace before the next section

% {\raggedright\textbf{``Khalifa university Baja SAE Project"}}\\
% I led and supervised a team of engineering students to design and produce a vehicle to compete in the SAE Baja Event during the summer of 2015. This project provides extensive hands-on experience on fabrication of various automotive parts using CNC technology, welding and conventional machining. Furthermore, I acquired a lot of experience throughout the design process on CAD modeling (SOLIDWORKS, Autodesk AutoCAD) and simulation (FEA Analysis of the Chassis using ANSYS mechanical APDL). The built design qualified for the Baja SAE 2015 annual competition in Maryland, United States, which was the university’s first undertaking in competing oversees. Being the team’s first encounter with oversees competitions, the vehicle passed rigorous technical inspection requirements and successfully completed all the events and the endurance race arriving at a rank of \textbf{58 out of a total of 100} competing international university teams.

% \programlang{\color{black}}{SOLIDWORKS, Autodesk AutoCAD}
% ~\programlang{\color{black}}{manufacturing}
% ~\programlang{\color{black}}{CNC machining}
% ~\programlang{\color{black}}{ANSYS mechanical APDL}

% \medskip % Extra whitespace before the next section

%----------------------------------------------------------------------------------------
%	PUBLICATIONS
%----------------------------------------------------------------------------------------

\section{Publications}

% Example \longformdescription{} command to add another publication:

%\longformpublication{Reference (format this manually as desired)}

%------------------------------------------------

% \textbf{\large Submitted preprints}\medskip \hrule \medskip

% \longformpublication{
% A. Khalil, \textbf{K. Al Handawi}, Z. Mohsen, A. Abdel Nour, R. Feghali, I. Chamseddine and M. Kokkolaras (2021). Predicting COVID-19 incidences from patients’ viral load using deep-learning. \textit{medRxiv}} \href{https://doi.org/10.1101/2021.08.14.21262064}{doi: 10.1101/2021.08.14.21262064}

%------------------------------------------------
\textbf{\large Submitted articles}\medskip \hrule \medskip

\longformpublication{\textbf{K. Al Handawi}, A. Brahma, D. Wynn, M. Kokkolaras and O. Isaksson (2023). Design space exploration and evaluation using margin-based trade-offs. \textit{Journal of Mechanical Design}\\
\textit{funded partially by NSERC and Area of Advance of Chalmers University}}

\textbf{\large Refereed Journal Articles}\medskip \hrule \medskip

\longformpublication{A. Khalil, \textbf{K. Al Handawi}, Z. Mohsen, A. Abdel Nour, R. Feghali, I. Chamseddine and M. Kokkolaras (2022). Weekly nowcasting of new COVID-19 cases using past viral load measurements. \textit{Viruses}, \textit{14}(7): pp 1414. \href{https://www.mdpi.com/1999-4915/14/7/1414}{doi: 10.3390/V14071414}}

\longformpublication{\textbf{K. Al Handawi} and M. Kokkolaras (2021). Optimization of infectious disease prevention and control policies using artificial life. \textit{IEEE Transactions on Emerging Topics in Computational Intelligence}, \href{https://doi.org/10.1109/TETCI.2021.3107496}{doi: 10.1109/TETCI.2021.3107496}
~\textit{funded by an NSERC discovery grant}}

\longformpublication{\textbf{K. Al Handawi}, M. Panarotto, P. Andersson, O. Isaksson and M. Kokkolaras (2021). Optimization of design margins allocation when making use of additive remanufacturing. \textit{Journal of Mechanical Design}, \textit{144}(1): pp 012001. \href{https://doi.org/10.1115/1.4051607}{doi: 10.1115/1.4051607}\\
\textit{funded partially by NSERC, FRQNT, CARIC and EU Horizon 2020 research and innovation programme}}

\longformpublication{M. Chehadeh, M. Wahbah, M. Awad, O. AbdulHay, \textbf{K. Al Handawi}, L. Seneviratne, I. Greatbatch and Y. Zweiri (2021). Novel aerial firefighting system for suppression of incipient cladding fires. \textit{Journal of Field Robotics}, \textit{1}: pp 203 -- 230. \href{https://doi.org/10.55417/fr.2021008}{https://doi.org/10.55417/fr.2021008}\\
\textit{funded by Emaar Properties PJSC}}


\longformpublication{\textbf{K. Al Handawi}, P. Andersson, M. Panarotto, O. Isaksson and M. Kokkolaras (2020). Scalable set-based design optimization and remanufacturing for meeting changing requirements. \textit{Journal of Mechanical Design}, \textit{143}(2): pp 021702. \href{http://dx.doi.org/10.1115/1.4047908}{doi: 10.1115/1.4047908}\\
\textit{funded partially by NSERC, FRQNT, CARIC and EU Horizon 2020 research and innovation programme}}

\longformpublication{\textbf{K. Al Handawi}, N. Vahdati, O. Shiryayev and L. Lawand (2017). Analytical modeling tool for design of hydrocarbon sensitive optical fibers. \textit{Sensors}, \textit{17}(10): pp 2227. \href{http://dx.doi.org/10.3390/s17102227}{doi: 10.3390/s17102227}\\
\textit{funded by Abu Dhabi National Oil Company}}

\longformpublication{L. Lawand, O. Shiryayev, \textbf{K. Al Handawi}, N. Vahdati and P. Rostron (2017). Corrosivity sensor for exposed pipelines based on wireless energy transfer. \textit{Sensors}, \textit{17}(6): pp 1238. \href{http://dx.doi.org/10.3390/s17061238}{doi: 10.3390/s17061238}\\
\textit{funded by Abu Dhabi National Oil Company}}

\longformpublication{\textbf{K. Al Handawi}, N. Vahdati, P. Rostron, L. Lawand and O. Shiryayev (2016). Strain-based FBG sensor for real-time corrosion rate monitoring in pre-stressed structures. \textit{Sensors and Actuators B: Chemical}, \textit{236}: pp 276 -- 285. \href{http://dx.doi.org/10.1016/j.snb.2016.05.167}{doi: 10.1016/j.snb.2016.05.167}\\
\textit{funded by Abu Dhabi National Oil Company}}

%------------------------------------------------

\textbf{\large Refereed Conference Papers}\medskip \hrule \medskip

\longformpublication{L. Lawand, T. Hajali, \textbf{K. Al Handawi} and A. Brahma (2023). Industrialisation of additive manufacturing: Assessing the impact of excess margins on manufacturing costs. \textit{in Proceedings of the 9th International Conference on Research Into Design}, Bengaluru, India, ICoRD'23.}

\longformpublication{\textbf{K. Al Handawi}, P. Andersson, M. Panarotto, O. Isaksson and M. Kokkolaras (2020). Scalable set-based design optimization and remanufacturing for meeting changing requirements. \textit{in Proceedings of the International Design Engineering Technical Conferences \& Computers and Information in Engineering Conference}, Virtual conference, IDETC2020.}

\longformpublication{L. Lawand, \textbf{K. Al Handawi}, M. Panarotto, P. Andersson, O. Isaksson and M. Kokkolaras (2019). A lifecycle cost-driven system dynamics approach for considering additive re-manufacturing or repair in aero-engine component design. \textit{in Proceedings of the Design Society: International Conference on Engineering Design}, Delft, Netherlands, ICED19: pp 1343 -- 1352. \href{http://dx.doi.org/10.1017/dsi.2019.140}{doi: 10.1017/dsi.2019.140}}

\longformpublication{\textbf{K. Al Handawi}, L. Lawand , P. Andersson, R. Brommesson, O. Isaksson and M. Kokkolaras (2018). Integrating additive manufacturing and repair strategies of aeroengine components in the computational multi-disciplinary engineering design process. \textit{in Proceedings of NordDesign}, Link\"{o}ping, Sweden, NordDesign 2018.}

\longformpublication{\textbf{K. Al Handawi}, N. Vahdati, O. Shiryayev, and L. Lawand (2016). Corrosion monitoring along infrastructures using distributed fiber optic sensing. \textit{in Proceedings of SPIE Smart Structures/NDE, International Society for Optics and Photonics, Sensors and Smart Structures Technologies for Civil, Mechanical, and Aerospace Systems}, Las Vegas, USA, SPIE2016. \href{http://dx.doi.org/10.1117/12.2218820}{doi: 10.1117/12.2218820}}

\longformpublication{L. Lawand, O. Shiryayev, \textbf{K. Al Handawi}, N. Vahdati and P. Rostron (2016). Corrosivity monitoring system using RFID-based sensors. \textit{in Proceedings of SPIE Smart Structures/NDE, International Society for Optics and Photonics, Sensors and Smart Structures Technologies for Civil, Mechanical, and Aerospace Systems}, Las Vegas, USA, SPIE2016. \href{http://dx.doi.org/10.1117/12.2218813}{doi: 10.1117/12.2218813}}

%------------------------------------------------

\textbf{\large Poster presentations}\medskip \hrule \medskip

\longformpublication{\textbf{K. Al Handawi}, P. Andersson, O. Isaksson and M. Kokkolaras (2019). Scalable set-based design solutions for product remanufacturing. \textit{International Conference on Engineering Design}, Delft, Netherlands, ICED19.}

\longformpublication{\textbf{K. Al Handawi}, L. Lawand, T. Hitchcox, Y. F. Zhao and M. Kokkolaras (2018). Additive manufacturing optimization and simulation platform for repairing and remanufacturing of aerospace components. \textit{CRIAQ RDV Forum}, Montr\'{e}al, Canada.}

%------------------------------------------------

\medskip % Extra whitespace before the next section


%----------------------------------------------------------------------------------------
%	INTERESTS
%----------------------------------------------------------------------------------------

\section{Research interests}

\columnratio{0.5,0.5} % The relative ratios of the three columns in the references

\setlength\columnsep{0.05\textwidth} % Specify the amount of space between the columns

\begin{paracol}{2} % Begin the multi-column environment

	\begin{itemize}\itemsep0em 
		\item Artificial intelligence in engineering design
		\item Design for changing requirements
		\item Robust design
		\item Reliability
		\item Numerical simulation
		\item Systems optimization
	\end{itemize}
		
	%----------------------------------------------------------------------------------------
	
	\switchcolumn % Switch to the next paracol column
	
	%----------------------------------------------------------------------------------------

	\begin{itemize}\itemsep0em 
		\item Surrogate modelling
		\item Stochastic programming
		\item Derivative-free optimization
		\item Computer aided design
		\item Computer aided engineering
		\item Manufacturing
	\end{itemize}

	%----------------------------------------------------------------------------------------
	
\end{paracol}

%----------------------------------------------------------------------------------------
%	SPECIAL ELECTIVE COURSES
%----------------------------------------------------------------------------------------

\section{Course work}

%------------------------------------------------

\columnratio{0.5,0.5} % The relative ratios of the three columns in the references

\setlength\columnsep{0.05\textwidth} % Specify the amount of space between the columns

\begin{paracol}{2} % Begin the multi-column environment

	\begin{itemize}\itemsep0em 
		\item Advanced mechanics of materials
		\item Engineering systems optimization
		\item Continuum mechanics
		\item Applied numerical methods
		\item Applied finite element analysis
	\end{itemize}
		
	%----------------------------------------------------------------------------------------
	
	\switchcolumn % Switch to the next paracol column
	
	%----------------------------------------------------------------------------------------

	\begin{itemize}\itemsep0em 
		\item Material engineering and corrosion
		\item Measurements and instrumentation
		\item Advanced vibrations
		\item Fracture mechanics	
		\item Viscous and compressible fluid flows
	\end{itemize}

	%----------------------------------------------------------------------------------------
	
\end{paracol}

%----------------------------------------------------------------------------------------
%	AWARDS
%----------------------------------------------------------------------------------------

\section{Awards and recognition}

% Example \tableentry{} command to add another line:

%\tableentry{Heading}{Content}{spaceafter}

% All 3 parameters must be supplied but any can be empty if you don't need them
% A "spaceafter" value in the third parameter will add some vertical space -- this is to be used between headings

%------------------------------------------------

\begin{supertabular}{>{\raggedleft\arraybackslash}p{0.17\linewidth}p{0.6\linewidth}>{\raggedleft\arraybackslash}p{0.17\linewidth}} % Start a table with three columns, the table will ensure everything is aligned
	
	%------------------------------------------------
	
	\tableentryrawthree{\textsc{May 2022 – Apr 2024}}{\textbf{Postdoctoral fellowship (PDF)}}{90,000 CAD}{spaceafter}
	\tableentryrawthree{}{\textit{National Sciences and Engineering Research council Canada}}{}{spaceafter}

	%------------------------------------------------
	
	\tableentryrawthree{\textsc{May 2019 – Dec 2021}}{\textbf{Doctoral Research award (B2X)}}{56,000 CAD}{spaceafter}
	\tableentryrawthree{}{\textit{Fonds de Recherche du Qu\'{e}bec - Nature et Technologies}}{}{spaceafter}
	
	%------------------------------------------------
	
	\tableentryrawthree{\textsc{Jan 2017 – Dec 2019}}{\textbf{McGill Engineering Doctoral Award (MEDA)}}{96,000 CAD}{spaceafter}
	\tableentryrawthree{}{\textit{McGill University}}{}{spaceafter}
	
	%------------------------------------------------

		
	\tableentryrawthree{\textsc{Aug 2013 – Dec 2015}}{\textbf{ADNOC Graduate fellowship}}{90,000 USD}{spaceafter}
	\tableentryrawthree{}{\textit{Abu Dhabi National Oil Company}}{}{}
	
	%------------------------------------------------
	
\end{supertabular}

%------------------------------------------------

\medskip \hrule \medskip

\begin{supertabular}{>{\raggedright\arraybackslash}p{0.77\linewidth}>{\raggedleft\arraybackslash}p{0.19\linewidth}} % Start a table with three columns, the table will ensure everything is aligned
	
	%------------------------------------------------
	
	\tableentryraw{Our \href{http://dx.doi.org/10.1115/1.4047908}{paper on scalable designs} was selected for the 2021 ASME Journal of Mechanical Design Editor's Choice award}{\textsc{ASME IDETC 2022, St.~Louis, USA}}{spaceafter}

	\tableentryraw{Awarded 2nd place for final problem presentation and winner of best data visualization in the 11th Montreal Industreal Problem Solving Workshop}{\textsc{IVADO, Universit\'{e} de Montr\'{e}al, Canada}}{spaceafter}

	\tableentryraw{Team leader of the first team to successfully qualify and complete the Baja SAE competition}{\textsc{Khalifa University, Abu Dhabi, UAE}}{spaceafter}

	\tableentryraw{Awarded 2nd place in the Abu Dhabi Solar Challenge (10,000 AED)}{\textsc{Khalifa University, Abu Dhabi, UAE}}{spaceafter}

	\tableentryraw{Recognition for voluntary commitment to the Graduate School's and the Graduate Student Affair's events}{\textsc{Khalifa University, Abu Dhabi, UAE}}{spaceafter}

	\tableentryraw{Graduated Honors with distinction (2,000 AED)}{\textsc{Khalifa University, Abu Dhabi, UAE}}{spaceafter}
	
	\tableentryraw{Made it to the Provost's list 3 times}{\textsc{Khalifa University, Abu Dhabi, UAE}}{spaceafter}
	
	%------------------------------------------------
	
\end{supertabular}

%----------------------------------------------------------------------------------------
%	REVIEW SERVICE
%----------------------------------------------------------------------------------------

\section{Review service}

%------------------------------------------------

\begin{supertabular}{>{\raggedright\arraybackslash}p{0.56\linewidth}>{\centering\arraybackslash}p{0.4\linewidth}} % Start a table with three columns, the table will ensure everything is aligned

	\textbf{Served as a reviewer on the following journals:}\\
	~\\

	%------------------------------------------------
	
	\tableentryraw{\centering\textbf{Journal}}{\textbf{Number of papers}}{} \hline

	\tableentryraw{Designs}{\textsc{1}}{}

	\tableentryraw{Scientific Reports}{\textsc{2}}{}

	\tableentryraw{Sensors and Actuators A}{\textsc{2}}{}

	\tableentryraw{IEEE Access}{\textsc{6}}{}

	\tableentryraw{Journal of Global Optimization}{\textsc{1}}{}
	
	\tableentryraw{Engineering Optimization}{\textsc{1}}{}
	
	\tableentryraw{Artificial Intelligence for Engineering Design, Analysis and Manufacturing}{\textsc{2}}{}
	
	\tableentryraw{AIAA Journal}{\textsc{2}}{}
	
	\tableentryraw{IEEE Transactions on Industrial Informatics}{\textsc{1}}{}
	
	\tableentryraw{Journal of Mechanical Design}{\textsc{1}}{}

	\tableentryraw{The Aeronautical Journal}{\textsc{3}}{}

	%------------------------------------------------
	
\end{supertabular}
%------------------------------------------------

\medskip % Extra whitespace before the next section

%----------------------------------------------------------------------------------------
%	COMPUTER SKILLS
%----------------------------------------------------------------------------------------

\section{Skills} 

% Example \tableentry{} command to add another line:

%\tableentry{Heading}{Content}{spaceafter}

% All 3 parameters must be supplied but any can be empty if you don't need them
% A "spaceafter" value in the third parameter will add some vertical space -- this is to be used between headings

%------------------------------------------------

\columnratio{0.5,0.5} % The relative ratios of the three columns in the references

\setlength\columnsep{0.05\textwidth} % Specify the amount of space between the columns

\begin{paracol}{2} % Begin the multi-column environment

	\begin{supertabular}{>{\raggedleft\arraybackslash}p{0.15\linewidth}l} % Start a table with three columns, the table will ensure everything is aligned

		%------------------------------------------------
		\tableentry{}{\Large\textsc{Programming Languages}}{spaceafter}
		\tableentry{}{\skilllevel{Python}{7}}{}
		\tableentry{}{\skilllevel{C++}{4}}{}
		\tableentry{}{\skilllevel{VB}{3}}{}
		\tableentry{}{\skilllevel{R}{5}}{}
		\tableentry{}{\skilllevel{MATLAB}{7}}{}
		\tableentry{}{\skilllevel{HTML, CSS}{3}}{}
		\tableentry{}{\skilllevel{Javascript}{2}}{}
		\tableentry{}{\skilllevel{mySQL}{5}}{}

		%------------------------------------------------
	
	\end{supertabular}
		
	%----------------------------------------------------------------------------------------
	
	\switchcolumn % Switch to the next paracol column
	
	%----------------------------------------------------------------------------------------

	\begin{supertabular}{>{\raggedleft\arraybackslash}p{0.15\linewidth}l} % Start a table with three columns, the table will ensure everything is aligned

		%------------------------------------------------
		\tableentry{}{\Large\textsc{Spoken Languages}}{spaceafter}
		\tableentry{English}{\skilllevel{Verbal}{9}}{}
		\tableentry{}{\skilllevel{Written}{8}}{}
		\tableentry{Arabic}{\skilllevel{Verbal}{9}}{}
		\tableentry{}{\skilllevel{Written}{8}}{}
		\tableentry{French}{\skilllevel{Verbal}{1}}{}
		\tableentry{}{\skilllevel{Written}{4}}{}
		\tableentry{Swedish}{\skilllevel{Verbal}{0}}{}
		\tableentry{}{\skilllevel{Written}{2}}{}
	
		%------------------------------------------------
	
	\end{supertabular}

	%----------------------------------------------------------------------------------------
	
\end{paracol}

\medskip % Extra whitespace before the next section

\begin{supertabular}{>{\raggedleft\arraybackslash}p{0.4\linewidth}>{\raggedright\arraybackslash}p{0.5\linewidth}} % Start a table with three columns, the table will ensure everything is aligned
	
	%------------------------------------------------
	
	\tableentry{Operating Systems ~~\faDesktop}{\faWindows ~~~\faLinux ~~~\faApple}{spaceafter}

	%------------------------------------------------

	\tableentry{Scientific libraries ~~\faCogs}{Qt, PyTorch, CUDA, Intel MPI, OpenCL, Pandas, scikit-learn}{spaceafter}

	%------------------------------------------------

	\tableentry{Source control ~~\faCodeFork}{Git, Perforce}{spaceafter}

	%------------------------------------------------

	\tableentry{Interactive development environments ~~\faListAlt}{VSCode, Xcode, Visual Studio, RStudio}{spaceafter}

	%------------------------------------------------
	
	\tableentry{Typesetting ~~\faFileText}{\LaTeX (and beamer), Microsoft Office}{spaceafter}
	
	%------------------------------------------------
	
	\tableentry{Finite element software ~~\faTh}{Ansys-APDL, Abaqus, NASTRAN}{spaceafter}

	%------------------------------------------------
	
	\tableentry{Application programming interfaces ~~\faCode}{Abaqus Fortran subroutines and python API, NX siemens API}{spaceafter}

	%------------------------------------------------
	
	\tableentry{CFD Software ~~\faSpaceShuttle}{Ansys (CFX, Fluent, Workbench) - basic usage}{spaceafter}

	%------------------------------------------------

	\tableentry{Computer aided design ~~\faCube}{SOLIDWORKS, NX siemens}{spaceafter}

	%------------------------------------------------

	\tableentry{Communication and interpersonal skills ~~\faUsers}{Excellent written and verbal presentation skills}{}
	\tableentry{}{Comfortable working in a target-driven and fast paced environment}{}
	\tableentry{}{Data analysis, proposal writing and questionnaire design}{}
	\tableentry{}{Very methodical and organized. Pays attention to detail and is able to identify underlying trends and patterns}{spaceafter}

\end{supertabular}
%------------------------------------------------

\medskip % Extra whitespace before the next section

%----------------------------------------------------------------------------------------
%	PERSONAL INTERESTS
%----------------------------------------------------------------------------------------

\section{Personal Interests}

\columnratio{0.5,0.5} % The relative ratios of the three columns in the references

\setlength\columnsep{0.05\textwidth} % Specify the amount of space between the columns

\begin{paracol}{2} % Begin the multi-column environment

	\begin{itemize}\itemsep0em 
		\item Gymnastics and Powerlifting training
		\item Automotive enthusiast
		\item Competitive gaming	
	\end{itemize}
		
	%----------------------------------------------------------------------------------------
	
	\switchcolumn % Switch to the next paracol column
	
	%----------------------------------------------------------------------------------------

	\begin{itemize}\itemsep0em 
		\item 3D printing hobbyist
		\item Tinkering and restoration of old machines
		\item Tabletop games (Warhammer and D\&D)
	\end{itemize}

	%----------------------------------------------------------------------------------------
	
\end{paracol}
%------------------------------------------------

\medskip % Extra whitespace before the next section

%----------------------------------------------------------------------------------------
%	REFERENCES
%----------------------------------------------------------------------------------------
\section{References}

%\textit{References available on request}

%------------------------------------------------

% Example \tableentry{} command to add another line:

%\tableentry{Heading}{Content}{spaceafter}

% All 3 parameters must be supplied but any can be empty if you don't need them
% A "spaceafter" value in the third parameter will add some vertical space -- this is to be used between headings

%------------------------------------------------

\columnratio{0.5,0.5} % The relative ratios of the three columns in the references

\setlength\columnsep{0.05\textwidth} % Specify the amount of space between the columns

\begin{paracol}{2} % Begin the multi-column environment

	\begin{supertabular}{rl} % Start a table with two columns, the table will ensure everything is aligned
	
		%------------------------------------------------
		
		\tableentry{}{\textbf{Prof. Michael Kokkolaras}}{spaceafter}
		\tableentry{Position}{Associate Professor}{}
		\tableentry{Employer}{\href{https://www.mcgill.ca/mecheng/}{Department of Mechanical Engineering}}{}
		\tableentry{}{\href{https://www.mcgill.ca/}{\textit{McGill Univeristy}}}{spaceafter}
		\tableentry{Email}{\href{mailto:michael.kokkolaras@mcgill.ca}{michael.kokkolaras@mcgill.ca}}{}
	
		%------------------------------------------------

		\tableentry{}{}{spaceafter}

		%------------------------------------------------

		\tableentry{}{\textbf{Prof. Fabian Bastin}}{spaceafter}
		\tableentry{Position}{Professor}{}
		\tableentry{Department}{\href{https://diro.umontreal.ca/accueil/}{Department of Computer Science}}{}
		\tableentry{}{\href{https://www.umontreal.ca/}{\textit{Universit\'{e} de Montr\'{e}al}}}{spaceafter}
		\tableentry{Email}{\href{mailto:fabian.bastin@umontreal.ca}{fabian.bastin@umontreal.ca}}{}
	
		%------------------------------------------------

		\tableentry{}{}{spaceafter}

		%------------------------------------------------

		% \tableentry{}{\textbf{Prof. Peter Rodgers}}{spaceafter}
		% \tableentry{Position}{Professor and Director, Research Laboratories}{}
		% \tableentry{Employer}{\href{https://www.ku.ac.ae/academics/college-of-engineering/department/department-of-mechanical-engineering}{Department of Mechanical Engineering}}{}
		% \tableentry{}{\href{https://www.ku.ac.ae/}{\textit{Khalifa Univeristy}}}{spaceafter}
		% \tableentry{Email}{\href{mailto:peter.rodgers@ku.ac.ae}{peter.rodgers@ku.ac.ae}}{}

		%------------------------------------------------

	\end{supertabular}

	%----------------------------------------------------------------------------------------
	
	\switchcolumn % Switch to the next paracol column
	
	%----------------------------------------------------------------------------------------

	\begin{supertabular}{rl} % Start a table with two columns, the table will ensure everything is aligned

	%----------------------------------------------------------------------------------------
	
		\tableentry{}{\textbf{Prof. Ola Isaksson}}{spaceafter}
		\tableentry{Position}{Professor}{}
		\tableentry{Employer}{\href{https://www.chalmers.se/en/departments/ims/}{Department of Industrial and Materials Science}}{}
		\tableentry{}{\href{https://www.chalmers.se/sv/Sidor/default.aspx}{\textit{Chalmers Univeristy of Technology}}}{spaceafter}
		\tableentry{Email}{\href{mailto:ola.isaksson@chalmers.se}{ola.isaksson@chalmers.se}}{}

		%------------------------------------------------

		\tableentry{}{}{spaceafter}
		
		%------------------------------------------------
		
		\tableentry{}{\textbf{Dr. Ibrahim Chamseddine}}{spaceafter}
		\tableentry{Position}{Instructor in Investigation}{}
		\tableentry{Department}{\href{https://gray.mgh.harvard.edu/}{Department of radiation oncology}}{}
		\tableentry{}{\href{https://hms.harvard.edu/}{\textit{Harvard Medical School}}}{spaceafter}
		\tableentry{Email}{\href{mailto:ichamseddine@mgh.harvard.edu}{ichamseddine@mgh.harvard.edu}}{}

		%------------------------------------------------

		\tableentry{}{}{spaceafter}

		%------------------------------------------------

		% \tableentry{}{\textbf{Prof. Nader Vahdati}}{spaceafter}
		% \tableentry{Position}{Associate Professor}{}
		% \tableentry{Employer}{\href{https://www.ku.ac.ae/academics/college-of-engineering/department/department-of-mechanical-engineering}{Department of Mechanical Engineering}}{}
		% \tableentry{}{\href{https://www.ku.ac.ae/}{\textit{Khalifa Univeristy}}}{spaceafter}
		% \tableentry{Email}{\href{mailto:nader.vahdati@ku.ac.ae}{nader.vahdati@ku.ac.ae}}{}
		
		%------------------------------------------------

		\tableentry{}{}{spaceafter}

		%------------------------------------------------

		% \tableentry{}{\textbf{Prof. Oleg Shiryayev}}{spaceafter}
		% \tableentry{Position}{Assistant Professor}{}
		% \tableentry{Employer}{\href{https://www.uaa.alaska.edu/academics/college-of-engineering/departments/mechanical-engineering}{Department of Mechanical Engineering}}{}
		% \tableentry{}{\href{https://www.uaa.alaska.edu/}{\textit{University of Alaska Anchorage}}}{spaceafter}
		% \tableentry{Email}{\href{mailto:oshiryayev@alaska.edu}{oshiryayev@alaska.edu}}{}

	\end{supertabular}

	%----------------------------------------------------------------------------------------
	
\end{paracol}

\medskip % Extra whitespace before the next section

%----------------------------------------------------------------------------------------

\end{document}